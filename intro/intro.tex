Distributed ledgers put forth a new paradigm for deploying online services
beyond the classical client-server model. In this new model, it is no
longer the responsibility of a single organization or a small consortium of
organizations to provide the platform for deploying relevant business
logic.
Instead, services can take advantage of decentralized, ``trustless'' computation
to improve their transparency and security as well as reduce the need for
trusted third parties and intermediaries.

\sloppy
Bitcoin~\cite{bitcoin}, the first successfully deployed distributed ledger protocol, 
 does not lend itself easily to the implementation of arbitrary 
 protocol logic that can support this paradigm.  
This led to many adaptations of the basic protocol for specific applications, such as NameCoin~\cite{namecoin}, a distributed domain
registration protocol, or Bitmessage~\cite{bitmessage}, a ledger-based communications
protocol. An obvious problem with this approach
is that, even though the Bitcoin
source code can be copied arbitrarily often, the Bitcoin community of software
developers and miners cannot, and hence such systems are typically
not sustainable. 
Smart contracts, originally posited as a form of reactive
computation~\cite{szabo1997formalizing}, were popularized by
Ethereum~\cite{ethereum}, solving these problems
by providing a uniform and standardized approach for deploying decentralized
computation over the same back-end infrastructure.

\fussy

Smart contract systems rely on a form of \emph{state-machine
  replication}~\cite{statemachinereplication}: All nodes involved in
maintaining the smart contract keep a local copy of its state, and advance this copy
with a sequence of requests. This sequence of requests needs to match for each
node in the system -- thus the need for consensus over which requests are made,
and their order. In practice, this is achieved through a distributed ledger.

A seemingly inherent limitation of the decentralized computation paradigm is the 
fact that protocol logic deployed as a smart contract has to be completely
non-private. This, naturally, is a major drawback for many
of the applications that can potentially take advantage of smart contracts. 
Promising cryptographic techniques for lifting this limitation
are zero-knowledge proofs~\cite{STOC:GolMicRac85}, and
secure-computation~\cite{STOC:GolMicWig87,Canetti02universallycomposable}.
Motivated by such cryptographic techniques, 
systems satisfying various definitions of privacy -- and requiring
various trust assumptions -- have been
proposed~\cite{zexe,SP:KMSWP16,enigmapaper,USENIX:KGCWF18}, as we detail in
\autoref{sec:relatedwork}. Their reliance on 
trust assumptions nevertheless fundamentally 
limits the level of decentralization which they can achieve, especially
  compared to their non-private
counterparts. 
For instance, a common restriction of such systems is to
assume a small, fixed set of participants at the core of the system.
This fundamentally clashes with the basic principles of a decentralized platform
like Bitcoin or Ethereum (collectively classified as {\em Nakamoto
  consensus}). In these systems, the set of parties maintaining the system 
can be arbitrarily large and independent of all platform performance parameters. 
This puts forth the following fundamental question that is the main motivation for our work. 

\begin{quote}\itshape
Is it feasible to achieve a \textbf{privacy-preserving} and \textbf{general-purpose}  
smart contract functionality under the same availability and decentralization 
characteristics exhibited by Nakamoto consensus? 
\end{quote}

In this work we carve out a large class of distributed computations that we
express as smart contracts, which we collectively refer to as ``\kachina\ core
contracts''. In particular, this includes contracts with privacy guarantees,
which can be implemented without additional trust assumptions beyond what is
assumed for Nakamoto consensus and the existence of a securely generated common
reference string.
The latter is not
an assumption to be taken lightly -- however it is a common requirement for
privacy-preserving blockchain protocols with strong cryptographic privacy
guarantees, and can be reduced to the same assumptions as the distributed
  consensus algorithm itself~\cite{pistis}. This
class allows us to express the protocol logic of dedicated privacy-preserving,
ledger-based protocols such as Zerocash~\cite{SP:BCGGMT14} as smart contracts.
Existing smart contract systems such as Zexe~\cite{zexe},
Hawk \cite{SP:KMSWP16}, Zether~\cite{zether}, Enigma~\cite{enigmapaper},
zkay~\cite{CCS:SBGMTV19}, and
Arbitrum~\cite{USENIX:KGCWF18} can be expressed, preserving their privacy
guarantees, as \kachina\ contracts. These protocols mainly rely on either
\emph{zero-knowledge} or \emph{signature authentication} for their security.
\kachina\ is flexible enough to allow contract authors to
express each of these systems, together with a concise description of the privacy
they afford. It does not supersede these protocols, but rather gives
 a common foundation on which one can build further privacy-preserving
systems.

%%% Local Variables:
%%% mode: latex
%%% TeX-master: "../main"
%%% End:
