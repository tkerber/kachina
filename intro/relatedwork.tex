\subsection{Related Work}
\label{sec:relatedwork}

There has been an increasing amount of research into smart contracts and their
privacy over the past few years. The results of these often focus on specific
use-cases or trust assumptions. We briefly discuss the most notable of
these.

\paragraph{Ethereum}

As the first practically deployed smart contract system,
Ethereum~\cite{ethereum} is the basis of a lot of our expectations and
assumptions about smart contracts. Ethereum is not designed for privacy, and
hides no data by itself. We assume that the reader is familiar with Ethereum.

\paragraph{Zexe}

Zerocash~\cite{SP:BCGGMT14} is a well-known privacy-preserving payment system,
allowing direct private payments on a public ledger. Zexe~\cite{zexe} extends
its expressiveness by allowing arbitrary scripts, reminiscent of
Bitcoin-scripts, to be evaluated in zero-knowledge in order to spend coin
outputs.
It is a major improvement in expressiveness over Zerocash, which only permits a
few types of transactions.

\paragraph{zkay}

zkay~\cite{CCS:SBGMTV19} extends Ethereum smart-contracts with types for private
data. It allows users to share encrypted data on-chain, and prove that data is
correctly encrypted and correctly used in subsequent interactions. These proofs
are managed through the ZoKrates~\cite{zokrates} framework, which compiles
Ethereum contracts into NIZK-friendly circuits. Its usage is limited to
fixed size pieces of private data.

\paragraph{Hawk}

One of the earliest works on privacy in smart contracts, Hawk~\cite{SP:KMSWP16}
is also one of the most general. It describes how to compile private variants
of smart contracts, given that all participants of the contract trust the same
party with its privacy. This party, the ``manager'', can break the contract's
privacy guarantees if they are corrupt, however they cannot break the correctness of the
contract's rules. The construction used in Hawk for the manager party relies of
zero-knowledge proofs of correct contract execution.

\paragraph{Zether}

\sloppy
A lot of work on privacy in smart contracts has focused on retro-fitting privacy
into existing systems. Zether~\cite{zether}, for instance, constructs a
privacy-preserving currency within Ethereum, which can be utilized for a number
of more private applications, such as hidden auctions. As with most retro-fitted
systems, Zether is constrained by the system it is built for, and does not
generalize to many applications.

\fussy

\paragraph{Enigma}

There are two forms of Enigma: A paper discussing running secure multi-party
computation for smart contracts~\cite{enigmapaper}, and a system of the same
name designed to use Intel's SGX enclave to guarantee
privacy~\cite{enigmawebsite}. The former has a lot of potential advantages, but
is severely limited by the efficiency of general-purpose MPC protocols. The
latter is a practical construction, and can claim much better performance than
any cryptography-based protocol. The most obvious drawbacks are the reliance on
an external trust assumption, and the poor track record of secure enclaves
against side-channel attacks~\cite{foreshadow}.

\paragraph{Arbitrum}

Using a committee-based approach, Arbitrum~\cite{USENIX:KGCWF18} describes how
to perform and agree on off-chain executions of smart contracts. A committee of
managers is charged with execution, and, in the optimistic case, simply posts
commitments to state updates on-chain. In the case of a dispute, an on-chain
protocol can resolve the dispute with a complexity logarithmic in the number of
computation steps taken. Arbitrum provides correctness guarantees even in the
case of a $n-1$ out of $n$ corrupt committee, however relies on a fully honest
committee for privacy.

\paragraph{State channels}

State channels, such as those discussed in~\cite{CCS:DziFauHos18}, occupy a
similar space to Arbitrum, due to their reliance on off-chain computation and
on-chain dispute resolution. The dispute resolution process is different, more
aggressively terminating the channel, and typically it considers only
participants on the channel that interact with each other. The privacy given is
almost co-incidental, due to the interaction being local and off-chain in the
optimistic case.

\paragraph{Piperine}

Piperine~\cite{piperine} uses a similar model and approach as presented in this
paper, relying on zero-knowledge proofs of correct state transitions, and
modeling smart contracts as replicated state machines. Piperine focuses on
efficiency gains from this approach, rather than privacy gains, which it does
not capture, while our work does not account for the benefit of transaction
batching. Our notion of state oracles can be seen as a generalization of the
state interactions presented in \cite{piperine}.

%%% Local Variables:
%%% mode: latex
%%% TeX-master: "../main"
%%% End:
