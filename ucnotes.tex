\section{UC Conventions}
\label{sec:conventions}

This paper is modeled and proven in the Universal Composability Framework
\cite{FOCS:Canetti01}, with global functionalities~\cite{TCC:CDPW07}. Although
we assume a basic familiarity with these concepts from the reader, the style of
writing UC protocols and functionalities may differ greatly from author to
author. As a result potentially important corner cases may be overlooked, as the
exact behavior of a given functionality is sometimes unclear. We adopt a more
explicit style (at least in our full functionalities), while at the same time
attempting to avoid writing unnecessary information in the definition of the
functionalities. While the proofs, protocols and functionalities can be read and
understood without explicit knowledge of the notation described in this section,
we define what behavior our notation leaves implicit in this section. Further, we
introduce some notations used throughout the paper that simplify our code,
however which are unrelated to UC.

\paragraph{Flow of execution}

Session identifiers are formally used in UC to shield a protocol from external
calls, except when allowed by the control function. While they are effectively a
technical detail of the description in UC, they are often replicated in the
description of functionalities and protocols. We leave all session identifiers
implicit instead. In a similar vein, it is often a convention to replicate (part
of) the input to a functionality when returning the result, to ensure that it is
clear which
query is being answered. We omit this as well, in favor of simply
stating the actual value returned. Both of these are how a protocol would be
written in a channel-based communications model, such as that of~\cite{cc}, rather
than the tape-based model of UC itself.

When a functionality is processing something, it is always processing \emph{on
  behalf of some party}, which may be the adversary itself, or may be corrupted.
Likewise when a protocol is processing something, it is processing this \emph{on
  behalf of its owning party}. When a functionality or protocol hands off
execution to another entity, by making a query to another functionality, or the
adversary, execution \emph{for this party} is suspended, and resumes only when
the query returns. Attempts by the environment to make queries to a suspended
protocol will be ignored. Likewise, if the environment attempts to query a
functionality with a party which is currently suspended, the query will also be
ignored. Crucially, the environment may still query a functionality with
\emph{another party} while one is suspended, ensuring that parties may still act
concurrently. We observe that this behaves equally in protocols and
functionalities, as the functionality is suspended in the same situation is the
corresponding party's protocol is suspended. Finally, we assume that queries
will eventually return -- this is equivalent to queries which do not return, as
we allow the environment and adversary to hold off indefinitely until returning.
While this is possible, in practice, due to the implicit suspension mechanism
described above, this means disabling a party permanently.

This above mechanism is not a great deviation from UC -- it can easily be
implemented by having a functionality or protocol record locally the suspension,
and reject new queries from the suspended party until it receives an input of a
specified form. We simply omit this mechanism when writing our protocols.
Something similar is in fact necessary for our security, as parties could
otherwise easily shoot themselves in the foot by concurrently creating
conflicting transactions. Responsive environments~\cite{AC:CEKKR16} are a
strictly stronger form of this idea.

We assume the existence of a set of all parties $\parties$, of which there is a
subset of honest parties $\honest \subseteq \parties$. We assume $\honest \neq
\varnothing$. Correspondingly, the set of corrupted parties is $\parties
\setminus \honest$. In this paper, static corruption is assumed, and all
functionalities are assumed to have knowledge of these sets. Real world
protocols, when they interact with these sets, will assume that, if the party
running them is $\party$, $\parties = \honest = \{\party\}$.

As a slight note, we use a somewhat unconventional model of having multiple
functionalities interacting with each other in the ideal world. This is largely
to avoid monolithic functionality design, using composition as a software design
primitive, rather than a security one.

\paragraph{Notation}

In terms of notation, we will explicitly declare and initialize all state
variables of functionalities and protocols, to make formal statements about them
more precise. For the same reason, the behavior of functionalities and
protocols is described via detailed pseudocode, instead of text. Most of the
notation is self-explanatory, however adversarial queries are not. As the
adversary may respond arbitrarily to queries, we include with each query a
well-formedness condition, and a fallback distribution. In particular, we write
\QueryAdv[y][P][D]{$x$}{} to mean the following: Send $x$ to $\adversary$, then
wait for the response $y$. If $P(y)$ does not hold, instead randomly sample $y$
from $D$. This allows us to ensure responses are well-formed, while avoiding the
common technique of aborting in the ideal world on receiving unexpected input,
something we try to avoid, as it effectively permits denial-of-service in the
``ideal'' world. Finally, we use the period (``$.$'') as a membership access
operator, to talk about variables of simulated functionalities, or in the proof,
to talk about state variables of various functionalities and protocol instances.
For instance, we write $\funct{}.X$ to mean the state variable $X$ within the
(possibly simulated) functionality $\funct{}$.

\paragraph{Functional Programming Constructs}

Besides understanding the message passing mechanics of UC, we assume only the
basic programming knowledge needed to read pseudo-code. We sometimes use
functional programming expressions, including the following for precision:

\begin{itemize}
  \item Lambda expressions:\\ $(\lambda x : 2x)(2) = 4$
  \item List and tuple literals:\\ $[1, 2]$, and $(1, 2)$.
  \item The higher-order function $\vn{filter}$:\\ $\vn{filter}(\lambda x: x
    \equiv 0 \mod 2, [1, 2]) = [2]$
\end{itemize}

We interpret maps as functions from keys to values. The symbols $\bot$ and
$\varnothing$ are overloaded, with the former representing both
``false'', and ``error/abort'', while the latter represents the empty set, empty
map, and in the case of contract states, the initial state (which the contract
itself may ascribe a different format to). Further, for a map $M$, we will write
$k \in M$ to mean that the map contains the key $k$. A key is not in the map iff
$M(k) \eqdef \bot$. For lists, we use $\epsilon$ to denote the empty list, and
$\concat$ to denote list concatenation. Single non-list items can be interpreted
as a singleton list.

We will use the following functions throughout the paper. $\vn{prefix}(L, x)$
returns the shortest prefix of $L$ containing $x$, or $L$ itself, if not such
prefix exists. $\vn{idx}(L, x)$ returns the index of the first occurrence of
$x$ in $L$, or $\bot$ if $x \notin L$. Finally, $\vn{dedup}(L)$ returns $L$
with any duplicate elements removed.

\begin{algorithmic}
  \Function{$\vn{prefix}$}{$L, x$}
    \State \Let $L' \gets \epsilon$
    \For{$y \loopin L$}
      \State $L' \gets L' \concat y$
      \If{$y = x$}
        \textbf{break}
      \EndIf
    \EndFor
    \State \Return $L'$
  \EndFunction
  \Function{$\vn{idx}$}{$L, x$}
    \State \Let $i \gets 0$
    \For{$y \loopin L$}
      \If{$y = x$}
        \Return $i$
      \EndIf
      \State \Let $i \gets i + 1$
    \EndFor
    \State \Return $\bot$
  \EndFunction
  \Function{$\vn{dedup}$}{$L$}
    \State \Let $L' \gets \epsilon$
    \For{$y \loopin L$}
      \If{$y \notin L'$}
        \Let $L' \gets L' \concat [y]$
      \EndIf
    \EndFor
    \State \Return $L'$
  \EndFunction
\end{algorithmic}

\noindent
Finally, we write $a \prec b$ for the list prefix relation, where we assume
reflexivity.

%%% Local Variables:
%%% mode: latex
%%% TeX-master: "main"
%%% End:
