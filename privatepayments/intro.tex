To demonstrate the versatility of \kachina, we take a closer look at the
(private) token contract, which is prone to the scalability issues \kachina\
addresses. Public token contracts are well understood, and
standardized~\cite{erc20}, with the typical implementation being to maintain a
mapping of ``addresses'' (hashes of public keys) to balances in the contract's
public state. We write the first \emph{provably private} token contract to
demonstrate the expressive power of \kachina.

A private token contract also implies that currency is not a primitive -- it can
be built as a contract, a key factor in simplifying our model, as it does
not need to encode currency as a special case. It provides an asset to
build contracts around in the first place, as well as a means of
denial-of-service mitigation, through transaction fees. Bad fee models have
resulted in devastating DoS attacks~\cite{ethereumdos},
highlighting their necessity.

We detail how to construct a fee model in
  \iffull\autoref{sec:fees}\else\cite[Appendix~J.5]{fullversion}\fi. The
  fundamental idea of this construction is to embed the transition function
  $\tnsfnkachina$ in a wrapper which performs the following steps:

\begin{enumerate}
  \item In the private state oracle, estimate the cost of transaction
    fees.
  \item Given an input gas price, and this estimate, pay these fees using a
    designated currency contract.
  \item Commit this as a partial execution success.
  \item Execute $\tnsfnkachina$ with a modified $\oracle_\sigma$, which deducts
    from available gas for each operation and aborts if this runs out.
  \item Transfer any remaining gas back to the transaction author.
\end{enumerate}

%%% Local Variables:
%%% mode: latex
%%% TeX-master: "../main"
%%% End:
