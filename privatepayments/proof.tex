\section{Proof Sketch of the Zerocash Contract}
\label{app:ppproof}

\begin{proof}[sketch, of \autoref{thm:zerocash}]
  To begin with, observe that from the collision resistance of PRFs,
  commitments, and sampling from $\set{0,1}^\secparam$, all coin commitments,
  serial numbers, and public keys will be unique with overwhelming probability.

  The environment can perform the following primary actions: a) For any honest
  party, run $\msg{post-query}{w}$. b) For any honest party, run
  $\msg{check-query}{\tx}$. c) For any party, run $\msg{submit}{\tx}$ against
  $\gl$. d) For any party, run $\msg{read}{}$ against $\gl$. e) Run
  $\msg{advance}{p, \ledgerstate'}$ against $\gl$, and f) for any party, run
  $\msg{extend}{\ledgerstate'}$ against $\gl$.
  
  All but the first two of these are trivial. The simulator forwards all queries
  to $\gl$, and the state of $\gl$ depends on no other functionality
  (transactions ``submitted'' in the ideal functionality are only passed to the
  adversary). As a direct result, the state and return value of $\gl$ follow the
  same distribution in both worlds, giving the environment no means to
  distinguish.

  During the running of $\msg{check-query}{}$ the environment does have a
  significant additional means of input, in the form of being able to assign
  meaning to adversarial transactions \emph{as they get executed for the first
  time}. It is sufficient to show the following: a) From the ideal-world
  leakage, the simulator can create indistinguishable real-world leakage. b)
  Ideal-world transactions have the same leakage descriptions sent to the
  environment (and are rejected under the same conditions). c) An invariant
  holds between the ideal and real-world contract state, such that it is
  preserved across both honest and adversarial transactions' transition function
  executions.

  We omit the full detail of this invariant. To sketch the idea behind it, we
  must prove that the following are preserved: The public keys recorded in the
  ideal contract state, and the simulator must correspond directly to secret
  keys recorded in the real contract state, and the same public keys returned by
  the real contract. Further, the coins held by honest parties in the real
  contract should be valid at any time, and correspond directly to the balance
  of the same party in the ideal contract. Honest unconfirmed transactions in
  both the real and ideal contracts should still be valid when they are finally
  executed (also implying they do not conflict with each other).

  These are preserved across honest $\msg{init}{}$ calls, as the simulator
  ensures the keys it stores, and the public keys returned in the ideal
  contract, are generated in the same way as in the real contract. They are
  preserved across honest $\msg{send}{}$ calls, as they remove one commitment
  from an honest party's coins, and potentially add it to the respective
  recipient party. Further, the leakage functions of honest $\msg{send}{}$s in
  the real contract ensures the same coin cannot be spent again. They are
  preserved across honest $\msg{mint}{}$s, as again the balance is incremented
  alongside a new coin being recorded. For adversarial transactions, as the
  simulator has all honest private keys, it can, and does, check if an honest
  party would register receiving a new coin. If a coin is sent, but no honest
  party receives it, the simulator records it as adversarial -- even if it may
  not be spendable by the real-world adversary. Further, the simulator manages
  which real-world coin commitments are associated with which adversarial public
  key in the ideal world. This ensures it can always spend a corresponding ideal
  coin to whatever was spent in the real world (assuming the real world
  adversary doesn't spend a coin he doesn't own, violating the one-wayness of
  the PRF).

  Transactions remaining valid in the ideal world is guaranteed by ensuring the
  balance of a party cannot fall below zero -- by assuming the worst case of
  only balance removing transactions becoming confirmed. Likewise, in the real
  world, the coins eligible for spending are those received in confirmed
  transactions, but not spent in unconfirmed ones, ensuring they will not
  conflict. In both cases, key generation will be refused if one is currently
  unconfirmed. $\msg{mint}{}$ and $\msg{balance}{}$ queries both only require
  initialization to have taken place in either world.

  To observe that the simulator creates indistinguishable leakage, we first note
  that the leakage for real-world $\msg{init}{}$ transactions is an empty
  transcript, which the simulator indeed recreates. For $\msg{send}{}$
  transactions, the simulator creates a public state transcript following the
  same structure of one in the real contract execution -- spending a coin,
  creating a new one, and sending a message. Here there are two cases: either
  the recipient is adversarial, or they are honest. In the case of an honest
  recipient, the simulator does not know the exact public key of the recipient.
  Fortunately, however, the environment does not know their secret keys for the
  same reason. As a result, it is sufficient to commit to an arbitrary coin, and
  encrypt arbitrary secrets. Due to the hiding of the commitments, and the
  key-privacy of the encryption scheme, the environment cannot distinguish this
  from a real transaction. The simulator creates a random serial number --
  revealing nothing due to collision resistance, and from the leakage of the
  length of the ledger, can reconstruct the corresponding Merkle tree root,
  revealing the same root as the corresponding real-world transaction.

  If the adversary \emph{is} the recipient, the simulator is given the actual
  public keys -- and can use these directly as in the real protocol, creating a
  valid spendable commitment, and a message the adversary can decrypt. Minting
  is similar to the case of sending to an honest party -- except no message is
  encrypted. For the same reason, the leakage is indistinguishable. Finally,
  honest balance queries have no leakage in the real world.

  For honest parties, the leakage descriptor the environment is asked to sign
  off on is identical -- for $\msg{init}{}$, $\msg{mint}{}$, and
  $\msg{balance}{}$ consisting of just this string, and for $\msg{send}{}$, it
  is $\msg{send}{t, \pk}$, where $\pk$ is the recipient, if it is adversarial,
  and otherwise is omitted. In each case, assertions made about the current and
  projected states are satisfied in either both worlds, or neither, ensuring the
  transaction is rejected or posted equally in both worlds. Specifically, all
  have tests for whether keys are initialized (asserting negatively in
  $\msg{init}{}$, and positively everywhere else). During spending, a positive
  spendable balance is also asserted in both worlds. These holding
  simultaneously is guaranteed by the invariant holding.

  Finally, the transaction outputs the environment receives are the same in both
  worlds: For $\msg{init}{}$, the simulator ensures it sees equally distributed
  public keys. For $\msg{balance}{}$, the equal distribution is guaranteed by
  the invariant. For all other messages, it will only see if they are in the
  ledger state -- as the honest transactions cannot fail, and return nothing.
  \qed
\end{proof}

%%% Local Variables:
%%% mode: latex
%%% TeX-master: "../main"
%%% End:
