\section{Liveness of Smart Contracts}
\label{sec:liveness}

We have presented \kachina\ in a model without liveness guarantees. In this
section, we discuss how to model, and prove \kachina\ secure in the presence of
such liveness guarantees. Similar principles can be used to argue for the
security of \kachina\ for further modifications to the ledger protocol, such as
adding consensus-level validation predicates.

\subsection{The $\delay$-delay Ledger}

While the simplified ledger $\gsimpleledger$ is nice for the analysis of
protocols building on it as a global functionality, in practice users would like
to take advantage of some liveness guarantees. We present a ledger
$\gdelayledger$, which annotates transactions with a time at which they were
received. This time is never returned to parties, however it asserts that every
party can see a transaction, once it is $\delay$ time slots in the past. This
ledger operates under the assumption of a global clock $\gclock$, which is also
presented here. We posit without proof that $\gdelayledger$ UC-emulates
$\gsimpleledger$, by virtue of the latter having far greater adversarial power.
We claim that this ledger can also be constructed using existing UC-secure ledgers
such as \cite{C:BMTZ17,CCS:BGKRZ18}, as these aim to provide the same
guarantees.

\subimport*{../uc/funct/}{delayledger}
\subimport*{../uc/funct/}{clock}

\subsection{Commutativity of Ledger Realizations}
\label{sec:commuc}

It is of note that since the ledger exists in both the ideal and real world, we
would ideally wish to be able to utilize the stronger $\delay$-delay ledger (and
others) in the ideal world as well. This is not trivial, however -- the UC proof
presented in this paper holds for the simple ledger, and while the transitivity
and composability of UC proofs implies that the simple ledger can be replaced by
the stronger ledger in the real world, this is not the only goal.

In order to enable ideal-world replacement, we consider when UC replacements are
commutative. Specifically, consider we have four functionalities, $\mathcal{A}$,
$\mathcal{B}$, $\mathcal{C}$, and $\mathcal{D}$, such that: a) $\mathcal{A}$ and
$\mathcal{B}$ both have $\mathcal{C}$ as a global functionality, b)
$\mathcal{A}$ is UC-emulated by $\mathcal{B}$ with the simulator
$\simul_{\mathcal{B}}$, and c) $\mathcal{C}$ is UC-emulated by $\mathcal{D}$
with the simulator $\simul_{\mathcal{D}}$. Observe that this is a generalization
of our situation, where $\mathcal{A}$ is $\fsc$, $\mathcal{B}$ is $\kachina$,
$\mathcal{C}$ is $\gsimpleledger$, and $\mathcal{D}$ is some other ledger $\gl$.

When can we conclude that $\mathcal{A}$ is still UC-emulated by $\mathcal{B}$
even if the global functionality is replaced by $\mathcal{D}$ in both worlds?
I.e. when can we perform the inner UC-replacement \emph{first}, and still be
able to perform the outer one?

\begin{theorem}
  Given $\mathcal{A}$, $\mathcal{B}$, $\mathcal{C}$, $\mathcal{D}$,
  $\simul_{\mathcal{B}}$, $\simul_{\mathcal{C}}$ as defined in
  \autoref{sec:commuc}, if $\simul_{\mathcal{B}}$ forwards all adversarial
  queries to $\mathcal{C}$ unchanged, and makes no queries to
  $\mathcal{C}$, then $\mathcal{A}$ is UC-emulated by $\mathcal{B}$ with the
  global functionality $\mathcal{D}$ in place of $\mathcal{C}$.
  \label{thm:commuc}
\end{theorem}

\begin{proof}
  We will provide this proof largely visually. The environment has a number of
  actions it can perform in any given world, in tandem with the dummy adversary.
  We will represent these as unconnected wires in a circuit representation of
  the different UC functionalities. Each wire is coloured in accordance with its
  purpose; these colours serve only to differentiate the wires. We visualize the
  preconditions of \autoref{thm:commuc} in \autoref{fig:commucbasicguc} and
  \autoref{fig:commucbasicuc}. This crucially includes the precondition that
  $\simul_{\mathcal{B}}$ forwards adversarial queries to $\mathcal{C}$, which is
  represented equivalently by these queries being made directly instead.

  \subimport*{../fig/}{commuctoplevel}
  \subimport*{../fig/}{commucbotlevel}

  By the UC emulation theorem, for all environments, executions with the
  simulator and the ideal-world protocol are equivalent to executions with the
  real-world protocol. Due to the all-quantification of the environment, we can
  replace any part of a circuit diagram which matches exactly one of the two
  sides of the equivalence with the other -- this is the foundation of the
  compositionality in UC.

  We first make use of this in the non-standard direction, of making our
  ideal-world protocol \emph{more ideal}. Specifically, replace the right-hand
  side of \autoref{fig:commucbasicuc} with the left. We start similarly to the
  left-hand side of \autoref{fig:commucbasicguc}, however using $\mathcal{D}$
  instead of $\mathcal{C}$. Visually, \autoref{fig:commucstep1} demonstrates the
  result, which includes two independent simulators.

  \subimport*{../fig/}{commucabstract}

  From here, we can realize both the ideal-world functionalities, provided it is
  in the correct order: We must first realize $\mathcal{A}$, as it relies on the
  presence of $\mathcal{C}$. We can directly apply the equivalence of
  \autoref{fig:commucbasicguc}, as can be seen in \autoref{fig:commucstep2}.

  \subimport*{../fig/}{commuctoprealise}

  Finally, nothing stands in the way of realizing $\mathcal{C}$ with
  $\mathcal{D}$, using the equivalence of \autoref{fig:commucbasicuc} again,
  this time in the more typical direction. As a result, we get in
  \autoref{fig:commucstep3} the final step, leading us to the intended
  equivalence, and proving the related UC-emulation statement.

  \subimport*{../fig/}{commucbotrealise}
  \qed
\end{proof}

As a final remark, we observe that $\kachina$ satisfies the preconditions for
commutativity, as $\simul_{\kachina}$ does not make adversarial queries to
$\gledger$.

%%% Local Variables:
%%% mode: latex
%%% TeX-master: "../main"
%%% End:
