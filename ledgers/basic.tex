\section{Ledger Functionalities}
\label{app:ledgers}

As smart contracts utilize an inderection layer of an underlying ledger, for
completeness we will define the behaviour of this underlying ledger. The key
characteristics of the ledgers used here are that they maintain a sequence of
transactions, which any user may submit new transactions to, which \emph{may}
then eventually appear in this sequence. The sequence itself is public, and can
be read by anyone.

We note in particular that for the purposes of this paper, many common features
of ledgers are not required -- in particular liveness is optional, although the
ideal-world guarantees for a ledger without liveness are naturally lessened.
Further, we do not specify a validation predicate, instead performing the
validation in our definition of smart contracts. This is solely to have a clean
separation between the consensus system and the semantics of transactions -- to
prevent denial-of-service attacks in a real system, this line would need to be
blurred, and transaction validation partially done in the consensus algorithm.

One feature typical ledger functionalities do not have, which we do explicitly
use here, is that of transactional privacy. Typically ledgers leak which party
created a new transaction, something we wish to avoid to strengthen the privacy
of our protocols. It is worth noting that this leakage is entirely network
based, and using a sender-anonymous network is sufficient to adapt ``leaky''
ledgers to this more private variant.

Finally, readers may be surprised that the ``private ledger'' of~\cite{SP:KKKZ19} is not
used. This is largely for the same reason that validation predicates are avoided
-- the separation of the ledger and transaction semantics is not possible in
this private ledger. While it would be possible to encode ideal smart contracts
as specific ledger blinding functions, this does not accurately model what we
intend to capture, and results in a large, monolithic, and hard-to-understand
functionality -- decidedly not ideal.

\subsection{The Perfect Ledger}

The perfect ledger is a platonic ideal form of a distributed ledger -- it allows
any party to instantly append to its record, and allows any party to read the
current sequence of recorded transactions. This ledger is very similar to that
used in~\cite{EC:KiaZhoZik16}, however it is not realized by existing distributed ledger
protocols -- not least as there is by necessity some network delay.

\subimport*{../uc/funct/}{perfectledger}

\subsection{The Simplified Practical Ledger}

The simplified ledger captures the essence of the traditional persistence
property of ledgers, although it does not capture liveness. Any user may post
transactions, which are deemed unconfirmed. The adversary may decide when and
which unconfirmed transactions to move to an append-only ledger, and may decide
how long a prefix of this ledger honest parties see -- provided it does not
remove anything previously revealed to them.

While the liveness property is not captured by this ledger, due to the large
amount of adversarial control, it is straightforward to see -- although we will
not demonstrate it here -- that more complex ledgers, such as those defined in
\cite{C:BMTZ17,CCS:BGKRZ18}, UC-emulate $\gsimpleledger$. In particular, this
means that if replaced in the ideal world with such a ledger, which \emph{does}
have the liveness property, we also in practice have liveness for our protocol.
We discuss the issue of liveness more in \autoref{sec:liveness}.

\subimport*{../uc/funct/}{simpleledger}

%%% Local Variables:
%%% mode: latex
%%% TeX-master: "../main"
%%% End:
