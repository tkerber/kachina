\subsection{Exporting Ledger Data}

Real-world smart contract systems often have some means to extract limited
information about the underlying consensus protocol, such as the hash of the
most recent block, the address of the block's miner, or the length of the
current chain. These can be useful in applications -- in particular the latter,
as it gives an provides an imprecise clock for use in contracts.

Clearly, these rely on tighter integration with the underlying consensus
mechanism than \kachina\ provides. We can still capture the core idea, by having
a sub-contract which manages such chain data, and allows this to be read and set
arbitrarily\footnote{This could be expanded to allow only certain types of
  setting -- such as advancing the time, but not rewinding it.}. We can then
assume that the correct usage of this sub-contract is enforced by the validation
of the underlying consensus mechanism -- transactions which attempt to
``incorrectly'' set the chain data -- for any definition of ``correct'' will
never reach the ledger.

\subimport*{../uc/tnsfn/}{chaindata}


The contract we present here does have a further issue: Since the loopback in
our multiplexers occurs only in the main transition function, the transcripts it
generates will commit to specific values for the ledger data upon transaction
creation -- something which is likely not reasonable. A more complex loopback
design, which we do not present here, would solve this: If calling into public
or private parts of other contracts were permitted from within the public and
private state oracles respectively.

For both leakage descriptors and dependencies, we make use of our assumption
that users cannot directly call $\msg{set}{}$.

\subimport*{../uc/desc/}{chaindata}
\subimport*{../uc/deps/}{chaindata}

%%% Local Variables:
%%% mode: latex
%%% TeX-master: "../main"
%%% End:
