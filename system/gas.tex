\subsection{Fees and Cost Models}
\label{sec:fees}

In order to prevent denial-of-service attacks, the computations performed by the
network in verifying a transaction must be paid for in some way. In public
currencies, there is typically a \emph{cost model}, which maps each step of
computation to a cost, often referred to as \emph{gas}. Each transaction declares
a limit on how much gas it is willing to pay, and what each unit's value should
be. It then pays the corresponding amount into a fees pool, and while executing
the transaction, the gas usage is counted. If the limit is reached, the
transaction is rejected, otherwise any spare gas is refunded.

We do not explicitly specify how miners are awarded these fees -- a simple
  approach is to not enable withdrawals from the fee pot within the transition
  function, relying on miners to do so themselves, and not include in their
  block other transactions which take from the pot.

In \kachina\ the computation done in public state oracles occupies a
  similar space: A modeling of fees must include estimating their likely cost,
  pay this estimation in advance, and then use up the gas during the actual
  oracle execution. In addition to this, the NIZK proof verification must be
  paid for. We will assume that this has a flat cost, dependant on the size of
  its inputs, i.e. the size of the transcript.

  Specifically, we assume two cost models: $\tnsfncostmodel$, and
  $\oraclecostmodel$, as well as a cost estimator $\oraclecostmodelest$.
  $\tnsfncostmodel$ is simply a function from a public state transcript to the gas
  cost of verifying a NIZK proof against it. A transaction will publicly declare
  what it believes the cost of its transcript is, and will use
  $\oraclecostmodelest$ (as well as a user input dictating the cost per unit of
  gas) to estimate the cost of the remaining transaction. The transaction
  declares this total fee, which part of the fee is for the NIZK verification,
  and what the cost per unit of gas is. Transactions which pay too little for
  NIZK verification, or set the cost per unit of gas too low, may not be
  picked up by miners, although modeling miner incentives is not within the
  scope of this paper.

  Formally, $g \gets \tnsfncostmodel(\transcript)$ is a function from a public
  state transcript to a gas cost, $(\sigma', g', y) \lor \bot \gets
  \oraclecostmodel(q, \sigma, g)$ is a function taking an oracle query, initial
  state, and gas limit, and either returning the result and remaining gas, or
  returning $\bot$ if the supplied gas ran out. Finally, $(\sigma', g', y) \gets
  \oraclecostmodelest(q, \sigma)$ returns an estimate as to the gas cost of
  running a query $q$, with the state $\sigma$ as a reference point.
  $\oraclecostmodelest$ and $\oraclecostmodel$ should return $(\sigma', y) =
  q(\sigma)$ if they succeed.

  In attaching fees to our contract system, we operate as follows:
  \begin{enumerate}
    \item In the private state oracle, simulate the transaction creation
      process, constructing the public state transcript $\transcript$, and for each public
      state interaction, recording the estimated cost, totalling to the overall
      gas cost $g$.
    \item For a given gas price $\vn{gasPrice}$, make two separate, public
      transfers into the fee pot: first $\tnsfncostmodel(\transcript) \times
      \vn{gasPrice}$, and second $g \times \vn{gasPrice}$
    \item Commit this as a partial execution success.
    \item Execute the transaction as normal, except making public state oracle
      queries through a modified gas cost oracle instead, retains a temporary
      state of the remaining gas.
    \item Finally, the public state oracle relinquishes the remaining gas, and
      returns it to the transaction creator.
  \end{enumerate}

It is worth noting that this is not entirely on par with existing fee models in
smart contract systems, due to one key issue: A failed transaction in \kachina\
has no effect by definition. This also means that no miner can be awarded gas
fees for a failed transaction, as the fee payment is itself an effect of the
transaction. There are a few ways this could be solved -- for instance
introducing multiple failure states with different effects, or splitting
transactions into linked fee-payment and contract call sub-transactions. For
instance, each transaction could consist of both a contract call transaction as
listed below, and a fee-payment transaction of the same value, with the
consensus rules ensuring the former is placed in the ledger iff the gas cost is
sufficient, and the latter otherwise. We do not construct these here, but
observe that in a practical system the payment of fees will require a special
status to combat denial of service attacks.

We now give an example transition function that combines this gas model with the
integrated payment system of \autoref{sec:payments}.

\subimport*{../uc/tnsfn/}{scs}
\subimport*{../uc/desc/}{scs}
\subimport*{../uc/deps/}{scs}

%%% Local Variables:
%%% mode: latex
%%% TeX-master: "../main"
%%% End:
