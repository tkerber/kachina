\section{Non-Atomic Executions}
\label{sec:nonatomic}

Smart contracts are typically closely linked with transactions made on the
underlying ledger, and indeed we explicitly make the same link in this paper.
That being said, there are numerous applications which do not rely on a single
transaction per interaction with a contract, from Hawk~\cite{SP:KMSWP16}, which
requires at least two transactions per round of interaction, to state
channels~\cite{CCS:DziFauHos18}, which have many of the same properties of smart
contracts, but may (under optimal conditions) not require transactions at all.

While the model of smart contracts presented in \autoref{sec:def}
technically excludes both of these, and a full treatment of both would require
further work, it is nonetheless worth considering how they can be -- albeit
imperfectly -- embedded in this model. First, let us consider contract queries
which require multiple on-chain interactions to ``complete''. As an example from
Hawk, consider Alice posts a query to a Hawk-style contract. Naturally, this
will not immediately return -- even if Alice's transaction has made it on-chain.
Instead, the transaction could return a ``future object'' -- a concept often
used in concurrent programming design, essentially just being a reference ID,
and a promise to associate some data with it later. Both Alice and the manager
party would have to regularly poll the contract -- e.g. send a contract query
$\msg{poll}{}$ every 10 minutes. On the manager's next $\msg{poll}{}$ query, he
would update the Hawk private state, and encrypt and post the result for Alice.
Finally, when Alice next polls, she would retrieve the result, and associate it
with the previous ``future object'' as an output. This sketches a protocol
running \emph{on top of \kachina}, which achieves this style of interaction.
It is worth noting that this requirement for end-users to interact is also a
limitation of the underlying model of universal composability: The environment
must manually instruct parties to resume, or messages to be forwarded by the
adversary.

In a similar vein, we can observe that some transactions need not be placed on a
ledger. In particular, if the shared, public state is not used, the transaction
is essentially ``off\-chain'', and there is no need to publicly post it. Further,
if the public state is used for message passing (such as in the construction of
a Zerocash contract above), this part of the transaction need not be on-chain --
sending an out of band message is cheaper. Using the same UC-based approach
described above, it would therefore be possible, for example, to first define an
ideal payment-channel contract, and prove that this is UC-emulated by a contract
implementing, for instance, Perun~\cite{SP:DEFM19}. Finally, we can argue that most
transactions in this contract can be omitted from the ledger, as they are just
two-party channel interactions. This is a rather roundabout means of
constructing off-chain communication, however it brings a crucial guarantee with
it, namely that it behaves the same under ledger reorderings as a purely
on-chain contract.

%%% Local Variables:
%%% mode: latex
%%% TeX-master: "main"
%%% End:
