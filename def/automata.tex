\subsection{Interactive Automata Interpretation}

Smart contracts are a form of \emph{reactive computation}: Parties supply an input
to the contract, the latter internally performs a stateful computation, and returns a
result to the original caller. The result is returned asynchronously, and may
depend on interactions with other users. This is quite close to the concept of a
trusted third party, although real-world systems have caveats:
\begin{itemize}
  \item They \emph{leak} information about the computation performed.
  \item They allow some \emph{adversarial influence}, partly due to relying on the
    transaction ordering of an underlying ledger.
  \item They may carry some impure \emph{execution context}: A transaction may
    depend on what the state is at the time it is created, for instance.
\end{itemize}

Often when talking about smart contracts, only the ``on-chain'' component
is considered. This is insufficient for privacy, as by its nature, everything
on-chain is public. We therefore model the off-chain component of the
interaction as well. This can be as simple as placing inputs directly on the
ledger, but can involve more complex pre-computation. Even without the need for
privacy, the need to model off-chain computation of smart contracts had been
observed~\cite{CKMM+19}, and we believe a formal model should account for it.

To represent a contract, we use a transition function, operating over the
contract's state. We denote the initial state as $\varnothing$. Transition
functions are deterministic, although limited nondeterminism can be simulated by
including randomness in the execution context. Notably, such randomness is
  fixed on transaction creation, allowing the creator to input (potentially
  biased) randomness, which is subsequently used in the (replicated) execution
  of the contract's state machine. Potential uses include the creation of
  randomized ciphertexts or commitments. The transition function will also
  output if a transaction should be considered ``confirmed'' or not, with the
  latter indicating failure or only partial success, which dependant
  transactions should not build on.

A contract \emph{transition function} $\tnsfn$ is a pure, deterministic
function with the format $(\jointstate', c, y) \gets \tnsfn(\jointstate, \party, w, z, a)$, with the
following inputs and outputs:
\vspace{-1em}
\begin{multicols}{2}
\begin{itemize}
  \item The current state $\jointstate$
  \item The calling party $\party$
  \item The input $w$
  \item The execution context $z$
  \item The adversarial input $a$
  \item The successor state $\jointstate'$
  \item The output $y$
  \item The confirmation state $c$
\end{itemize}
\end{multicols}
\vspace{-1em}

In addition to the transition function, it is necessary to capture what leakage
an interaction with the contract has. The two are separated due to the
asynchronous nature of smart contracts -- a transaction is made, and leaks
information, before the corresponding transition function is run on-chain.

The leakage is captured by a \emph{leakage function}, which receives the same
input, and further receives the creating user $\party$'s ``view'' $\partyview$ of the
contract as an input. $\partyview = (\ell, U_\party, T, \jointstate)$ consists of four items: a) The
length of $\party$'s view of the ledger $\ell$. b) $\party$'s unconfirmed
transactions $U_\party$. c) A map $T$ from $\tx \in U_\party$ to $(\party, w,
z,\allowbreak a, D)$. These are $\tnsfn$'s inputs, and the transaction's dependencies, which we will introduce
shortly, $D$. d) The contract's state according to $\party$'s view of the ledger,
$\jointstate$. This ``view'' may be used to avoid attempting
double-spends by selecting a coin to spend which no other unconfirmed
transaction uses, for instance. For this purpose the leakage
function can also abort by returning $\bot$, refusing to create a transaction.
The function returns a leakage value $\vn{lkg}$, which is passed to the
adversary, a description of the leakage which occurred, $\vn{desc}$, a list
of transactions to depend on, $D$, and the context $z$. While $\vn{lkg}$ may be
arbitrary, it is important that $\vn{desc}$ provides an accurate and readable
description of this leakage. Its primary purpose is to allow parties to decide
\emph{not} to go ahead with a transaction if they notice the leakage is more
than expected. With complex contracts, anticipating what will be leaked should
not be relied upon. The usage of a descriptor highlights that
  $\lkgfn$ should not be maliciously supplied, and facilitates simulation, as
  shown in \autoref{app:privatepayments}.
The leakage may additionally depend on the current time, a list of
  unconfirmed transactions and -- for each transaction -- their corresponding
  inputs to $\tnsfn$ and their dependencies. This may be used to avoid
  attempting double-spends for instance, by ensuring that the context specifies
  to use a coin which no other unconfirmed transaction uses.

It is worth emphasising that the leakage discussed in this paper is
  deliberate; this is not leakage observed over a network, which can be hard to
  identify, but is instead information which users accept to reveal. For
  instance, a leakage in Zerocash~\cite{SP:BCGGMT14} is the length of the ledger at the time a
  transaction is created, with the security of the protocol guaranteeing that
  this -- but nothing more -- is revealed to an adversary.

The list of dependencies $D$ is a list of transactions, which must occur
in the same order before the newly created transaction can be applied. This can
be used to enforce basic ordering constraints between transactions. Finally, the
context $z$ allows information about the state at the time of transaction
creation to be passed to the transition function. This may include the current
state, unconfirmed transactions, and a source of randomness. Its content is left
arbitrary at this point.

A \emph{leakage function} $\lkgfn$ is a pure, non-deterministic
function with the format $(\vn{desc}, \vn{lkg}, D, z) \gets
\lkgfn(\partyview,\party, w)$, with the following inputs and outputs:
\vspace{-1em}
\begin{multicols}{2}
\begin{itemize}
  \item $\party$'s contract view $\partyview$
  \item The calling party $\party$
  \item The input $w$
  \item The leaked data $\vn{lkg}$
  \item The leakage descriptor $\vn{desc}$
  \item The tx dependencies $D$
  \item The context $z$
\end{itemize}
\end{multicols}
\vspace{-1em}

We consider the pair $(\tnsfn, \lkgfn)$ to define a smart contract, and
  parameterize the ideal smart contract functionality presenting in
  \autoref{sec:fsc} by this pair. The ideal
world interaction with a smart contract follows the below pattern:

\begin{enumerate}
  \item A party submits a contract input $w$.
  \item The corresponding context and leakage are computed.
  \item The party agrees to the leakage description, or cancels (in the latter
    case, the transaction never takes place, and no information is revealed).
  \item The adversary is given $(\vn{lkg}, D)$, and provides the adversarial
    input $a$.
  \item The submitting party can retrieve the output of
    $\tnsfn$ (if any), while other parties can interact with the modified state.
\end{enumerate}

The level of privacy guaranteed depends greatly on the leakage function
$\lkgfn$: A leakage function which returns its input directly as leakage
provides no privacy, while one which returns no leakage at all provides almost total
privacy (notably the fact some interaction was made is still leaked). By tuning
this, the privacy of Ethereum, Zerocash, and everything in between can be
captured.

Our model relies on users querying the result of transactions manually -- they
are not notified of the acceptance of a transaction, and can not modify it once
made. If a transaction is not yet confirmed by the ledger, the user gets the
result $\msg{not-found}{}$, if the transaction depends on failed transactions,
$\bot$ is returned, and otherwise the result is provided by the
contract itself (which may also inform of partial success).

%%% Local Variables:
%%% mode: latex
%%% TeX-master: "../main"
%%% End:
