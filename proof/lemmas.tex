\subsection{Supporting Lemmas}

Both $\fsc$ and $\kachina$ have $\vn{exec}$ functions, which executes an entire
ledger state given to it. \autoref{lem:execpreserve} is a generalization of
invariant \ref{inv:execcons}, and simply states that this execution will
preserve the invariant, and return the same values in the real and ideal world.

\begin{lemma}
  For any $\party \in \honest, \ledgerstate$ where $\ledgerstate \prec
  \gl^r.\ledgerstate \lor \gl^r.\ledgerstate \prec \ledgerstate$, and after
  sending the message $\msg{extend}{\ledgerstate \setminus \gl^r.\ledgerstate}$ to $\gl$,
  $((\sigma^i, \boldsymbol\rho^i), y^i, C^i)$ is the result of running
  $\vn{exec}(\ledgerstate)$ in $\fsc$, and $((\sigma^r, \rho^r),
  y^r, C^r)$ is the result of running $\vn{exec}(\ledgerstate)$ in
  $\phi_\party$, these interactions preserve $\boldsymbol I$, and the returned
  values are equivalent: $\sigma^i = \sigma^r \land \boldsymbol\rho^i[\party] =
  \rho^r$. If the last transaction $\tx$ in $\ledgerstate$ is owned by $\party$
  (i.e. $\fsc.T(\tx) = (\party, \ldots)$), then $y^i = y^r$,
  otherwise $y^r = \bot$.
  \label{lem:execpreserve}
\end{lemma}

\begin{proof}
  First, we consider the $\msg{extend}{}$ call. This will only extend if
  $\ledgerstate$ is longer than $\gl.\ledgerstate$ -- otherwise it extends with
  $\epsilon$, which is a no-op. This call preserves $\boldsymbol I$, as
  demonstrated in \autoref{sec:proof}.

  We prove by induction over $\ledgerstate$. In the base case, $\ledgerstate =
  \epsilon$. The invariant is trivially preserved, and the returned values are
  equivalent (when $\varnothing$ is interpreted as public/private state pairs).
  In the induction step, we proceed by case analysis for the new transaction
  $\tx = (\transcript, D, \nizkproof)$:

  \paragraph{Case 1} The $\tx \in \fsc.T$, and all processed transactions so
  far have been also been recorded (are in $\fsc.T$). If so, then by
  \ref{inv:execcons}, the return values are equivalent. Further, this iteration
  does not change the state in the ideal world. By \ref{inv:honestwit} and
  \ref{inv:rejected}, we also know that the transaction is either in
  $\fnizkr.\proofs$, or $\fnizkr.\disproofs$. As a result, no state changes will
  be made in the real-world execution either, trivially preserving $\boldsymbol
  I$.

  \paragraph{Case 2} $\tx \notin \fsc.T$, but $((\transcript,
  D), \nizkproof) \in \fnizkr.\disproofs$. In this case, the real
  world will skip this transaction, and set $y$ to $\bot$. In the ideal world,
  the simulator will ensure that $\fsc.T(\tx)$ is set to
  $\textsc{none}$, and equally this transaction is skipped, with $y$ set to
  $\bot$. This affects and preserves the following invariants:

  \begin{enumerate}
    \item[\ref{inv:wit}] As by \ref{inv:iswit}, $\tx$ has no witness.
    \item[\ref{inv:honestwit}] As $\fsc.T(\tx) =
      \textsc{none}$, not satisfying the precondition.
    \item[\ref{inv:honestrec}] As $\tx$ was not in $\fsc.T$ in
      the induction hypothesis, and is not associated with an honest party.
    \item[\ref{inv:anyrec}] By $\fsc.T(\tx)$ being
      $\textsc{none}$, satisfying the postcondition.
    \item[\ref{inv:rejected}] Due to $((\transcript, D), \nizkproof) \in
      \fnizkr.\disproofs$.
    \item[\ref{inv:unseenunconfirmed}] As $\fsc.T(\tx) =
      \textsc{none}$, not satisfying the precondition.
    \item[\ref{inv:results}] As $\tx$ was not in $\fsc.T$ in the induction
      hypothesis, it cannot be in any $\phi_\party.Y$, by \ref{inv:honestrec}.
    \item[\ref{inv:execcons}] By the output equivalence part of the induction
      step holding.
    \item[\ref{inv:postunrecorded}] By $\tx$ being previously unrecorded,
      further restricting the quantification domain, and $\fsc.T(\tx) =
      \textsc{none}$, not satisfying the precondition.
    \item[\ref{inv:recordedunconf}] By the newly recorded transaction being in
      the ledger state, as this has been extended if necessary.
  \end{enumerate}

  \paragraph{Case 3} $\tx \notin \fsc.T$, but $((\transcript, D), \nizkproof)
  \in \fnizkr.\proofs$. In this case, by \ref{inv:iswit} a witness must be
  recorded, and by \ref{inv:wit} this witness must be accessible to the
  simulator. As a result, the simulator will ensure that $T(\tx)$ is set to
  $(\adversary, w, (\transcript_\sigma, \oracle(\transcript_\rho)), \varnothing,\allowbreak
  D)$. As this is an adversarial transactions, the $\rho$-value of the adversary
  is not constrained, and neither is the output $y$-value. As a result, to show
  the execution equivalence holds, it suffices to show that both worlds will
  have the same $\sigma$-value after this new transaction is the same in both
  worlds.
  In the real world, the commit-separated form of $\transcript_\sigma$ is
    applied to $\sigma$ in parts, with the last non-$\bot$ state being adopted.
    In the ideal world, the $\transcript_\sigma$ is recomputed, and the parts
    compared with those passed as inputs. The confirmation depth is derived from
    how many parts match before the computed and input transcripts diverge, or
    the result is $\bot$. The ideal world runs $\runtnsfn(\sigma, \transcript_\sigma', w,
    \oracle(\transcript_\rho), \bot)$. Since $((\transcript_\sigma, D), (\transcript_\rho, w)) \in \lang$
  (by \ref{inv:iswit}), we know that the public state oracle in $\runtnsfn$ can
  be replaced with $\oracle(\transcript_\sigma)$, up to the confirmation depth,
  after which the executions may diverge. As a result, the $\sigma$ returned in
  the ideal world -- $\partstate{\sigma}$ indexed at the confirmation depth --
  matches that returned in the real world.\ 
  As $\fsc.T$ is set, the following parts of the invariant are
  affected and preserved:

  \begin{enumerate}
    \item[\ref{inv:wit}] By the left hand side of the disjunction already being
      satisfied.
    \item[\ref{inv:honestwit}] By the transaction being recorded in the NIZK,
      and the simulator knowing its witness.
    \item[\ref{inv:honestrec}] As $\tx$ was not in $\fsc.T$ in
      the induction hypothesis, and is not associated with an honest party.
    \item[\ref{inv:anyrec}] By the newly recorded transaction satisfying the
      postcondition.
    \item[\ref{inv:rejected}] By the newly recorded transaction not being
      recorded as rejected.
    \item[\ref{inv:unseenunconfirmed}] By the newly recorded transaction not
      being honestly owned, not satisfying the precondition.
    \item[\ref{inv:results}] As $\tx$ was not in $\fsc.T$ in the induction
      hypothesis, it cannot be in any $\phi_\party.Y$, by \ref{inv:honestrec}.
    \item[\ref{inv:execcons}] By the output equivalence part of the induction
      step holding.
    \item[\ref{inv:postunrecorded}] By $\tx$ being previously unrecorded,
      further restricting the quantification domain, and $\fsc.T(\tx) =
      (\adversary, \ldots)$, not satisfying the precondition.
    \item[\ref{inv:recordedunconf}] By the newly recorded transaction being in
      the ledger state, as this has been extended if necessary.
  \end{enumerate}

  \paragraph{Case 4} The transaction has not been previously seen -- that is
  $\tx \notin \fsc.T$, and $((\transcript, D), \nizkproof) \notin
  \fnizkr.\proofs \cup \fnizkr.\disproofs$. In this case, both the
  real and ideal worlds will attempt the same NIZK verification (simulated in
  the ideal world). By \ref{inv:proofs}, they will both query the adversary
  for a NIZK witness in the same way, handing off execution. By the induction
  hypothesis, $\boldsymbol I$ holds as the point of execution transfer, and as
  the query made is the same in both worlds, the environment gains no means to
  distinguish.

  As NIZK verification is the first thing done in both worlds, and NIZK
  verification is agnostic as to which party is verifying, this is equivalent to
  the environment \emph{first} manually verifying the same statement/proof pair.
  As will be shown in \autoref{sec:proof}, this preserves the invariant, and
  returns the same result in both worlds. Therefore, Case~4 is equivalent to
  either Case~2 (if the NIZK verification failed), or Case~3 (if the NIZK
  verification succeeded), as if the NIZK verification is done externally
  beforehand, the statement/proof pair \emph{must} be either in
  $\fnizkr.\proofs$, or in $\fnizkr.\disproofs$.

  \paragraph{Case 5} $\tx \in \fsc.T$, however
  \ref{inv:execcons} cannot be applied, as other transactions have since been
  added. By \ref{inv:postunrecorded}, we know that $\tx$ belongs to an honest
  party $\party'$, and that $\tx \in \phi_{\party'}.U$. We will use \ref{inv:dep}
  to argue that, where $(\transcript_\rho, z) = \phi_{\party'}.T(\tx)$,
  either $\transcript_\rho(\boldsymbol\rho^i[\party'], z) \neq \bot$, or the
  transaction is skipped in both worlds.

  First, we consider the possibility that $\tx \notin \phi_{\party'}.U$. By
  \ref{inv:unseenunconfirmed} we know that $\tx$ cannot ever be confirmed by a
  suffix of the ledger state referred to in the invariant. As this is a prefix
  of $\gl^r.\ledgerstate$, such that it contains no unrecorded transactions, the
  current induction is necessarily a suffix of it. As a result, we know that the
  ideal world execution will fail. As transactions are rejected in both worlds
  under the same conditions -- due to dependencies not being satisfied
  -- we can conclude that these transactions
  are also skipped in the real world, preserving $\boldsymbol I$ as no state
  variables are changed, and satisfying all conditions by the induction
  hypothesis. We will now focus on the case that $\tx \in \phi_{\party'}.U$

  Next, we determine that, given $\tx \in \phi_{\party'}.U$, the longest prefix
  $\ledgerstate^*$ referred to in \ref{inv:dep} is a prefix of the ledger state
  $\ledgerstate$ we are currently performing induction over. We know it to be a
  prefix of $\gl.M(\party')$, such that this prefix contains none of the
  transactions in $\phi_{\party'}.U$. As $\tx \in \phi_{\party'}.U$, and is
  either a prefix or extension of $\gl^r.\ledgerstate$, of which
  $\gl^r.M(\party')$ is itself a prefix by \ref{inv:ledgerord}, we can conclude
  that $\ledgerstate^* \prec \ledgerstate$.

  To apply \ref{inv:dep}, we are only concerned with the parties private
  state $\rho$, we can observe all transactions in $\ledgerstate \setminus
  \ledgerstate^*$ are either not owned by $\party'$, will not be accepted in any
  context, or are in $\phi_{\party'}.U$. We can ignore the first possibility, as
  the real world execution of them will not affect $\rho$, regardless. The
  second can also be ignored, as these will be skipped by the ideal world
  execution, and by induction hypothesis, by the real world execution as well.
  Next, we consider which of the transaction in $\ledgerstate \setminus
  \ledgerstate^*$ owned by $\party'$ have been successfully processed.
  $\vn{exec}$ provides replay protection, ensuring that each unconfirmed
  transaction has been processed at most once. By induction hypothesis, the
  sequence $A$ of such transactions that have results associated for this
  party is the same in both worlds.
  As $\vn{exec}$ will not set the state to $\bot$, we know that there
    exists a confirmation depth vector $\confirmations$, such that
  $\transcript^*_{ES(\vn{map}(\phi_{\party'}.T, A), \confirmations)}(\boldsymbol\rho^*[\party'])
  \neq \bot$ is the result of applying these transactions in the real world. Here,
  $\boldsymbol\rho^*$ is taken to be the private state
  corresponding to the prefix $\ledgerstate^*$ in the ideal world.

  Now that $\tx$ is processed, we know by \ref{inv:dep} that its dependencies
  are either in confirmed, and in order, in $\ledgerstate^*$, or in
  $\phi_{\party'}.U$. In either case, $\tx$ is skipped in both worlds if it is a
  replayed transaction, or if would set $\transcript_\sigma(\sigma) = \bot$. In
  the ideal world, additionally it gets skipped if it would set the contract
  state -- either $\sigma^i$ or $\boldsymbol\rho^i$ -- to $\bot$.

  As $\tx \in \phi_{\party'}.U$, and is not a replay, $B = A \concat \tx$ is a
  permutation of a subset of $\phi_{\party'}.U$. As a result, by \ref{inv:dep},
  we know that $\transcript^*_{ES(\vn{map}(\phi_{\party'}.T,
    B), \confirmations \concat \cdot)}(\boldsymbol\rho^*[\party']) \neq \bot$. As we've previously established
  this holds for $A$, by definition of $\transcript^*$, this implies that\ 
  applying $\transcript_\rho$ to $\boldsymbol\rho[\party']^i$ to any
    confirmation depth is non-$\bot$, where
  $\boldsymbol\rho[\party']^{i}$ is the same as the ideal world private state for
  the induction hypothesis, by repeated application of \ref{inv:results}.
  Likewise, \ref{inv:results} allows us to conclude that $\sigma^i$ will also be
  non-$\bot$, as the update applied to it will be equivalent to applying
    $\transcript_\sigma$ to the same confirmation depth, which by definition of
    confirmation depth is not $\bot$.
  
  If the transaction is skipped in both worlds, the induction hypothesis still
  applies. Otherwise, up to the confirmation depth, applying both public
    and private state transcript parts is non-$\bot$. As previously
  noted in \autoref{sec:oracles}, this is equivalent to partial oracle
    executions to this confirmation depth, and therefore the ideal and real
    world states match.\ Likewise, \ref{inv:results} applies (as by \ref{inv:honestrec},
  $\phi_{\party'}.Y(\tx)$ is set), and we know the ideal-world result $y^i =
  \phi_{\party'}.Y(\tx)[c]$, where $c$ is the
    confirmation depth.

  As in the real world $\sigma^r \gets \transcript_\sigma(\sigma^r)$, their
  equality is preserved.\ If $\party = \party'$, by \ref{inv:honestrec},
  $\phi_\party.T$ is defined, and as a result, the same update is carried out to
  $\rho$ in the real world, as to $\boldsymbol\rho^i[\party]$ in the ideal
  world. Further, it will return the same result $y^r$ as the ideal world, as
  $\vn{proj}_c(\phi_\party.Y(\tx)) = y^i$. If $\party \neq \party'$, the ideal world update
  does not affect $\boldsymbol\rho^i[\party]$, and the correctness of the
  returned private state is guaranteed by the induction hypothesis. $y^r = \bot$
  is returned, which satisfies the requirements. Finally, in both cases,
  if the confirmation depth is maximal, the transaction is added to $C$, ensuring the returned $C$ is
  the same in both worlds. Neither world makes any state updates, trivially
  preserving $\boldsymbol I$.
  \qed
\end{proof}

\begin{lemma}
  \sloppy
  If $\boldsymbol I$ holds, then for all $\party \in \honest$, running
  $\vn{updateState}(\party)$ in $\fsc$, and running $\vn{updateState}$
  in $\phi_\party$ preserves $\boldsymbol I$.
  \label{lem:updatestate}
\end{lemma}

\fussy
\begin{proof}
  To begin with, both worlds retrieve the same value $\ledgerstate$ /
  $\ledgerstate_\party$ from $\gl$, due to \ref{inv:ledger}. As seen in
  \autoref{sec:proof}, this preserves $\boldsymbol I$. Next, by
  \autoref{lem:execpreserve}, both worlds receive the same value $C$, and the
  $\vn{execConfirmed}$ call preserves $\boldsymbol I$. The worlds now iterate over
  $\phi_\party.U$ and $\fsc.U_\party$ respectively, which by
  \ref{inv:unconfirmed} are equal in value. The operations performed are almost
  identical, with the exception of the real world deconstructing $u = (\cdot, D,
  \cdot)$ for each $u \in U$, while the ideal world extracts $(\ldots, D) =
  \fsc.T(u)$ instead. By \ref{inv:anyrec}, if $u \in \fsc.T$, the two are
  equivalent, and by \ref{inv:honestrec}, as $u \in \phi_\party.U$, it is also
  in both $\phi_\party.T$, and $\fsc.T$. We conclude that both worlds perform
  the same operations. Updated is only $\phi_\party.U$ and $\fsc.U_\party$
  respectively. The following parts of the invariant are affected and preserved:

  \begin{enumerate}
    \item[\ref{inv:honestrec}] By reducing the scope of $\phi_\party.U$.
    \item[\ref{inv:dep}] This consists of three sub-parts: The satisfaction of
      $J$, that of $\vn{sat}$, and that $D \setminus C \setminus U =
      \varnothing$. The first is trivial: $J$ makes a statement about all
      permutations of subsets. A smaller initial set merely reduces the scope of
      the quantifiers. The second holds due to $\vn{updateState}$ ensuring that
      if a transaction is removed from $U$, any transactions that depend on it
      are also removed, with the remaining transactions being in the same order
      as before. As a result, a previously satisfied transaction is either
      removed itself, or still satisfied, as it does not depend on any removed
      transactions, and dependencies still in $U$ being in the same order as
      before. Finally, $D \setminus C \setminus U = \varnothing$ is also
      preserved due to the recursive removal. Specifically, if $D \nsubseteq (C
      \cup U)$ the corresponding transaction is removed. As a result, only
      transactions satisfying this condition will remain.
    \item[\ref{inv:unseenunconfirmed}] As the removed transactions either fail
      confirmation directly (it depends on a transaction rejected in
      $\ledgerstate_\party$, or a different transaction order than got
      enforced), or depends on a transaction which fails. In either case, any
      state $\ledgerstate'$, of which $\ledgerstate_\party$ is a prefix, cannot
      accept these for the same reasons.
    \item[\ref{inv:postunrecorded}] As in \ref{inv:unseenunconfirmed}.
    \item[\ref{inv:unconfirmed}] By equal update.
  \qed
  \end{enumerate}
\end{proof}

%%% Local Variables:
%%% mode: latex
%%% TeX-master: "../main"
%%% End:
