\subimport*{./}{theorem}

We prove \autoref{thm:kachina} through a detailed case-analysis of any action an
environment, in conjunction with the dummy adversary, may take. The full case
analysis may be found in \iffull\autoref{sec:fullproof}\else\cite[Appendix~D]{fullversion}\fi. We define an invariant $\boldsymbol I$ between the real and
ideal executions in the UC security statement, roughly encoding that ``the real
and ideal states are equivalent''. This ranges from simple equivalences, such as
them having the same ledger states, or the same NIZK proofs considered valid, to
complex invariants, such as all unconfirmed honest transactions satisfying the
sub-invariant $J$ of \autoref{def:invdep}. This invariant is used to argue that the
environment, in combination with a dummy adversary, cannot distinguishing
between the real and ideal worlds. Specifically, for any action the environment
takes, $\boldsymbol I$ is preserved, and from $\boldsymbol I$ holding, we can
conclude that the information revealed to it, or the dummy adversary, is
insufficient to distinguish the two worlds.

The simulator for \kachina\ is quite straightforward; it simply creates
simulated NIZK proofs for all honest transactions, and forces the adversary to
reveal witnesses to the simulated NIZK functionality in time for these to be
input to the ideal smart contract. Fundamentally, the security proof relies on
state transcripts being interchangeable with full state oracles in the same
setting, and this setting being enforced by both the protocol and functionality.

While a lot of factors must be formally considered, this is derived from
receiving NIZK proofs as part of valid transactions, which prove precisely that
if the preconditions for the transaction are met, then the update performed on
the public state is the same. The private state is a little more tricky, but is
guaranteed by the dependency invariant $J$ holding for honest parties. This lets
us similarly argue that the private state transcript will have the same effect
as the ideal-world execution.

%%% Local Variables:
%%% mode: latex
%%% TeX-master: "../main"
%%% End:
