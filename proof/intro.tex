If an environment can distinguish between the ideal and real executions in
presence of our simulator (see \autoref{sec:simulator}), then there must exist
some polynomial sequence of interactions permitting it to distinguish with a
non-negligible advantage. Broadly, each of the environment's actions fall into
one of three categories: a) Honestly interacting with the protocol. b) Honestly
interacting with the ledger. c) Commanding the adversary to perform some action
in the real world. We will consider the responses the environment makes to
queries given to the dummy adversary separately, in each case at the point where
the query is made.

We will consider in parallel two random variables of the state of the ideal
world execution, and that of the real world execution at any time. We leave out
of our analysis the ``stack'' of partial executions (as described in
\autoref{sec:conventions}), except to show that the flow of each party -- i.e.
when it is waiting for which query to be answered -- is the same in both worlds.
In particular, the state of the ideal world has the following
functionalities' states as a part of it: 1. the state of the simulator,
$\simul$, 2. the state of the smart contract functionality, $\fsc$, and finally
3. the state of the ledger $\gl^i$. In the real world, for each $\party \in
\honest$, $\party$'s protocol state, which we refer to as $\phi_\party$, is part
of the state, along with the (shared) NIZK hybrid functionality $\fnizkr$, and
the real-world ledger $\gl^r$. For convenience, we will often talk about these
states as concrete variables, and not random variables.

We will prove inductively that any action the environment takes will do two
things: First, it will preserve an invariant $\boldsymbol I$, which holds after
the state of both worlds at any point during the two experiments. Second, if the
invariant holds, the environment gains at most negligible advantage in
distinguishing from its next action. To begin, we will specify the simulator,
the invariant $\boldsymbol I$, followed by a few lemmas helpful in the proof.
Finally, we will perform the induction itself.

%%% Local Variables:
%%% mode: latex
%%% TeX-master: "../main"
%%% End:
