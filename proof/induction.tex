\subsection{Proof of \autoref{thm:kachina}}
\label{sec:proof}

We proceed with the main inductive proof of \autoref{thm:kachina}. We consider
the base case of the system initializations in the real and ideal worlds. The
induction hypothesis is that after $k < 2^\secparam$ interactions with any
environment, the state of both worlds satisfy the invariant $\boldsymbol I$, and
the environment has not gained a non-negligible advantage in distinguishing. We
will assume, without loss of generality, the adversary being a dummy adversary.
We provide a concrete list of actions the environment may take before taking the
induction step. We note that as at any point the environment cannot distinguish,
we can assume that it takes the same action in both worlds without loss of
generality.

\paragraph{Base Case}

\begin{proof}
  Most base cases hold either due to equal initialization of variables
  constrained to be equal, or due to initialization leaving forall quantifiers
  to quantify over the empty set. The former is the case for:
  \ref{inv:ledger}, \ref{inv:proofs}, \ref{inv:honestrec}, and
  \ref{inv:unconfirmed}. The latter is the case for: \ref{inv:wit},
  \ref{inv:honestwit}, \ref{inv:anyrec}, \ref{inv:rejected},
  \ref{inv:unseenunconfirmed}, \ref{inv:results},
  \ref{inv:postunrecorded}, and \ref{inv:recordedunconf}. The remaining hold
  for the following reasons:

  \begin{enumerate}
  \item[\ref{inv:dep}] \sloppy At initialization, the only prefix of $\gl^r.M(\party)$
    is $\epsilon$. $\phi_\party.\vn{execState}(\epsilon) = (\varnothing,
    \varnothing)$. The base case therefore holds iff $J(\varnothing,
    \varnothing)$ holds. This in turn holds iff
    $\transcript^*_\epsilon(\varnothing) \neq \bot$, or $\varnothing \neq \bot$.
  \item[\ref{inv:execcons}] \fussy At initialization, the only ledger state
    $\ledgerstate$ which satisfies the condition that $\forall \tx \in
    \ledgerstate : \fsc.T(\tx) \neq \bot$ is $\epsilon$. For
    this, as both worlds are initialized to equivalent contract states, the
    outputs of $\vn{exec}$ will be equal.
  \item[\ref{inv:ledgerord}] By the reflexivity of $\prec$.
  \qed
  \end{enumerate}
\end{proof}

\paragraph{Induction Step}

\begin{proof}
We observe that the environment is capable of the following queries:

\begin{itemize}
\item $\forall \party \in \honest, w$:\footnote{We omit without loss of
    generality the environments ability to make honest queries with corrupted
    parties. The environment may simulate running the honest protocol to
    replicate these.} Sending $\msg{post-query}{w}$ to $\fsc$ or
  $\allowbreak\kachina$.
\item $\forall \party \in \honest, \tx$: Sending $\msg{check-query}{\tx}$ to
  $\fsc$ or $\allowbreak\kachina$.
\item $\forall \party \in \parties, \tx$: Sending $\msg{submit}{\tx}$ to $\gl$.
\item $\forall \party \in \parties$: Sending $\msg{read}{}$ to $\gl$.
\item $\forall \Sigma'$: Sending $\msg{extend}{\Sigma'}$ to $\gl$.
\item $\forall \party, \Sigma'$: Sending $\msg{advance}{\party, \Sigma'}$ to
  $\gl$.
\item $\forall \party \in \parties \setminus \honest$: Sending $\msg{prove}{x,
    w}$ to $\fnizk$.
\item $\forall \party \in \parties$:\footnote{Technically, as in
    $\msg{prove}{}$, the environment can only instruct corrupted parties to
    verify. As verification for honest parties preserves the invariant as well,
    and is a useful lemma, we prove the more general statement.} Sending
  $\msg{verify}{x, \nizkproof}$ to $\fnizk$.
\end{itemize}

We will prove that $\boldsymbol I$ is preserved across any of these queries, and
that they reveal the same information in both worlds.

\paragraph{Case {\normalfont$\msg{post-query}{w}$}}

We proceed by sub-case analysis. We identify the following cases: 1. The
transaction is rejected by the contract. 2. The transaction is rejected by the
user. 3. The transaction is posted. In all cases, $\vn{updateState}$ is first
run. By \autoref{lem:updatestate}, this preserves the invariant, and also
ensures that the returned value $\ledgerstate_\party = \gl.M(\party)$ is the
value returned in both worlds (by \ref{inv:ledger}). In the ideal world,
$\lkgfn$ is called. The real world largely emulates the same, computing most of
the same values identically. Of note are the values $\sigma^o$ and
$\rho^o$/$\boldsymbol\rho^o[\party]$, which are computed in both worlds using
$\vn{execState}(\ledgerstate_\party)$. By \autoref{lem:execpreserve}, this
preserves the invariant, and returns the same values.

The only place where the two worlds diverge in their computation is in handling
the unconfirmed transactions -- the ideal world executes $\runtnsfn$, and
updates $\sigma^\pi$, $\rho^\pi$, and $X$ according to the confirmation
  depth, while the real world partially applies $\transcript_\sigma$ and
  $\transcript_\rho$ to the confirmation depth. Before we go into the
main three cases, we will argue that, if the transaction is \emph{not} rejected
by the contract, then these two approaches will yield the same result, and that
they will reject equally.

To begin with, in the ideal world the confirmation depth is derived from the
number of transcript parts matching between the newly generated and input
transcripts. As a transcript application is non-$\bot$ if and only if it can be
generated in the same way in the current state, this ensures that the
confirmation depth matches in the two worlds.

Further, we observe that in the real world, the final value of $\rho^\pi$
\emph{cannot} be $\bot$ -- to begin with, $\vn{updateState}$ guarantees that
$\ledgerstate_\party \cap \phi_\party.U = \varnothing$. This in turn, along with
\ref{inv:dep} ensures that $J(X)$ holds, as well as that $\vn{sat}^{*}(X, U)$ holds.
It follows $\transcript^*(\rho^o) \neq \bot$, where $\transcript^*$ performs the
same repeated applications of $\transcript_\rho(\rho, z)$ as the loop in the
main protocol, using the same values. Further, by \ref{inv:wit} and
\ref{inv:honestwit}, we can conclude that the transcripts $\transcript_\rho$,
and contexts $z$ are the same in both worlds. By \ref{inv:results} , we can
conclude that the final $\rho^\pi$ values are also the same in both worlds.
Further, as $\transcript_\rho$ and $z$ are equal in both worlds, and by
\ref{inv:anyrec} $D$ is also equal, the sequence $X$ is also equal.
Subsequently, $\sigma$, $\rho$, and $D$ are computed equivalently in both
worlds.

We now consider the main case analysis: If the contract rejects the transaction
in the ideal world, the returned description is $\bot$. This happens if and only
if $\vn{last}(\partstate{\sigma}) = \bot \lor \vn{last}(\partstate{\rho}) = \bot$, the same condition as the real world
protocol has for rejecting the transaction before querying the user.\ If the
transaction is rejected, no variables are modified, preserving $\boldsymbol I$,
and the same value is returned in both worlds, giving the environment no means
to distinguish.

If the contract does not automatically reject the query, the leakage descriptor
is computed equally in both worlds, and sent to the party to acknowledge. The
party has the opportunity to accept the described leakage, or cancel the
transaction. At the point of handing over execution to the environment, no state
has been modified, trivially preserving $\boldsymbol I$, and as the same leakage
descriptor is given, it has no means to distinguishing.

In the case of the environment subsequently cancelling the transaction, both
worlds immediately return with $\msg{rejected}{}$, again trivially preserving
$\boldsymbol I$, and giving no means to distinguish.

Finally, if the environment accepts the leakage, both worlds obtain the
transaction identifier $\tx$: The simulator ensures that the real-world
adversary is queried for the same NIZK proof as it is in the real-world, and
that the transaction format matches that of real-world transactions. At the time
of the proof query, no state has been modified, trivially preserving
$\boldsymbol I$. As the same statement is queried for, the environment gains no
information to distinguish.

Subsequently, both worlds record the transaction's information (in $\fsc.T$ and
$\phi_\party.T$), and note is as unconfirmed (in $\fsc.U$ and $\phi_\party.U$).
In the real world, the result is further recorded in $\phi_\party.Y$. The
following parts of $\boldsymbol I$ are affected and preserved (including the
$\msg{prove}{}$ query):

\begin{enumerate}
  \item[\ref{inv:proofs}] By $((\transcript_\sigma, D), \nizkproof)$ being
    added to both worlds' $\proofs$ equally.
  \item[\ref{inv:wit}] As $\phi_\party.T$, $\fsc.T$, and
    $\phi_\party.Y$ are appropriately set to satisfy the RHS of the disjunction.
  \item[\ref{inv:honestwit}] As for the newly added transaction, $\party \in
    \honest$, and $((\transcript_\sigma, D),\allowbreak \nizkproof) \notin
    \simul.\fnizk.W$ (by the uniqueness of statement/proof pairs).
  \item[\ref{inv:honestrec}] As the newly added transaction is added to all of
    $\phi_\party.T$, $\phi_\party.U$, and $\fsc.T$, where it is
    associated with $\party$.
  \item[\ref{inv:anyrec}] As the newly added transaction does consist of
    transcript, dependencies and proof, and the former two are recorded in
    $\fsc.T$ correctly, and $\simul$ returns $\varnothing$ for
    $a$.
  \item[\ref{inv:rejected}] As the newly recorded transaction is not recorded
    as $\textsc{none}$.
  \item[\ref{inv:dep}] By $J$ being preserved when appending a new
    transaction, $J$ holds after the induction step (as $\rho$ remains unaffected).
    $\vn{sat}$ holds by induction hypothesis, and as $D \sqsubseteq U
    \setminus \set{\tx}$. For the new transaction, $D \setminus C \setminus \phi_\party.U =
    \varnothing$ as $D \subseteq \phi_\party.U$; for previously transactions this still
    holds, as $\phi_\party.U$ is expanded.
  \item[\ref{inv:unseenunconfirmed}] As the newly added transaction is also
    unconfirmed by the owning party.
  \item[\ref{inv:results}] By $\transcript_\sigma$, $\transcript_\rho$, and $\partresults$
    having been extracted from $\runtnsfn$, operating in the context of $w$,
    $z$, and some $\sigma$, $\rho$, with these values being recorded in the
    corresponding state variables (except $\sigma$ and $\rho$). As the
    transcripts are transcripts of oracle evaluations against $w$, and $\partresults$ is
    the result of $\tnsfnkachina$ operating with these oracles, executing
    $\tnsfnkachina_{\oracle(\transcript_\sigma), \oracle(\transcript_\rho)}(w)$ has
    the same effect. Further, as the transcripts accurately reproduce the state
    change in the original state context, by definition of transcript
    execution, if the sequence of transcripts up to the confirmation
      depth can be applied to be non-$\bot$, they are indistinguishable from
      making the original queries to the state oracle. Combined with the sequence
    of queries made depending only on $w$ and the state oracle itself, we can
    conclude that the transcript applications are the same as executing against
    the state oracles up to the confirmation depth, regardless of which initial state the transcript could be
    successfully applied to.
  \item[\ref{inv:execcons}] As a new transaction has been recorded, we must
    now additionally consider transaction sequences $\ledgerstate$ which contain
    this new transaction at some point. We cannot directly use
    \autoref{lem:execpreserve}, however we can make use of its induction: If
    we can show that any $\ledgerstate$ \emph{ending} with the new transaction
    $\tx$ satisfies the execution equivalences, then induction from
    \autoref{lem:execpreserve} can apply on that as a base case (in
    particular, the precondition for Case~5 applies for all subsequent
    transactions). The execution equivalence holds for this new base case, as we
    know that this new transaction is both honest, and considered unconfirmed
    for this party. Therefore, the argument for Case~5 holds for $\tx$ itself as
    well. As the execution equivalence defined in \autoref{lem:execpreserve}
    is the same as that of \ref{inv:execcons}, this part of the invariant is
    preserved.
  \item[\ref{inv:postunrecorded}] By the newly added transaction being
    unconfirmed, it satisfies any quantification where $\tx$ is set to it. By
    now being recorded, the range of quantifications for $\tx'$ is restricted,
    relaxing the condition.
  \item[\ref{inv:unconfirmed}] By equal update.
  \item[\ref{inv:recordedunconf}] By the newly recorded transaction being
    considered unconfirmed by an honest party.
\end{enumerate}

Finally, both worlds submit to the ledger the same transaction $\tx$, which
simply sent to the adversary. At this point $\boldsymbol I$ holds as argued
above, and as the same transaction is sent, the environment cannot distinguish.
Finally, $\msg{posted}{\tx}$ is returned, giving the environment no information
to distinguish for the same reasons.

\paragraph{Case {\normalfont$\msg{check-query}{\tx}$}}

After running $\vn{updateState}$, \autoref{lem:updatestate} preserves the
invariant, but also ensures that $\ledgerstate_\party = \gl.M(\party)$, where
$\ledgerstate_\party$ is the value returned in both worlds (by
\ref{inv:ledger}).

We consider three cases: 1. $\tx \notin \ledgerstate_\party$, 2. $\fsc.T(\tx) =
(\party, \ldots)$, and 3. otherwise. In Case~1, both worlds return
$\allowbreak\msg{not-found}{}$ without updating any state, not allowing the environment to
distinguish, and preserving $\boldsymbol I$. In Case~2, both worlds run
$\vn{execResult}(\vn{prefix}(\ledgerstate_\party, \tx))$, preserving $\boldsymbol I$
according to \autoref{lem:execpreserve}, and returning the same value in both
worlds, giving the environment no information to distinguish. Finally, in
Case~3, only the real world runs $\vn{execResult}$, while the ideal world returns
$\bot$. As previously in $\vn{updateState}$ the sub-function $\vn{execConfirmed}$ was
run, we know that all NIZK-verifications performed in this $\vn{exec}$ call have
previously been made -- as a result the call modifies no state, and preserved
$\boldsymbol I$. Further, by \autoref{lem:execpreserve}, it returns $\bot$, as
in the ideal world, giving the environment no information to distinguish.

\paragraph{Case {\normalfont$\msg{submit}{\tx}$}}

In both worlds $\tx$ is handed to the adversary, and no other action is taken.
As the same information is relayed, the environment cannot distinguish, and as
no state is changed, $\boldsymbol I$ is preserved.

\paragraph{Case {\normalfont$\msg{read}{}$}}

By \ref{inv:ledger}, both worlds will return the same result; therefore the
environment cannot distinguish. As no state is changed, $\boldsymbol I$ is
preserved.

\paragraph{Case {\normalfont$\msg{extend}{\Sigma'}$}}

As nothing is returned, the environment gains no information allowing it to
distinguish. By \ref{inv:ledger}, the updates done are the same in both
worlds. The parts of the invariant affected and preserved are the following:

\begin{enumerate}
  \item[\ref{inv:ledger}] By equal update.
  \item[\ref{inv:execcons}] Extending $\gl.\ledgerstate$ further constrains the
    possible $\ledgerstate$ values quantified over.
  \item[\ref{inv:postunrecorded}] Without loss of generality, we can assume
    single-trans\-action appends to $\gl^r.\ledgerstate$. If a new unrecorded
    transaction is added, it (at first) does not preceed any transactions,
    leaving the quantification unchanged, and relaxing the non-existance
    quantifier. If a recorded honest transaction is added, then by
    \ref{inv:unseenunconfirmed} and \ref{inv:ledgerord}, this transaction
    satisfies the conditions.
  \item[\ref{inv:ledgerord}] By the append-only nature of $\msg{extend}{}$.
  \item[\ref{inv:recordedunconf}] By relaxing the constraint.
\end{enumerate}

\paragraph{Case {\normalfont$\msg{advance}{\party, \Sigma'}$}}

As nothing is returned, the environment gains no information allowing it to
distinguish. By \ref{inv:ledger}, the updates done are the same in both
worlds. The parts of the invariant affected and preserved are the following:

\begin{enumerate}
  \item[\ref{inv:ledger}] By equal update.
  \item[\ref{inv:dep}] Without loss of generality, we can assume
    single-trans\-action advances. If $\gl.M(\party) \cap \phi_\party.U \neq
    \varnothing$, or the newly added transaction $\tx \in \phi_\party.U$, this
    is preserved as the longest prefix remains equal. Otherwise, $\ledgerstate$
    in the induction step is that of the induction hypothesis, with one
    transaction $\tx \notin \phi_\party.U$ appended. If $\tx$ is not owned by
    $\party$, by \ref{inv:honestrec}, $\phi_\party.T(\tx) = \bot$, and
    therefore $\vn{execState}(\ledgerstate \concat \tx)$ returns the same
    $\rho$ as $\vn{execState}(\ledgerstate)$, preserving the invariant. If
    $\tx$ is owned by $\party$, by \ref{inv:unseenunconfirmed}, this
    transaction will be rejected, likewise returning the same $\rho$. Further,
    as $D \setminus C \setminus U$ is already $\varnothing$ for all dependency
    lists $D$, and extending $\ledgerstate$ can only lead to $C$ growing, this
    condition remains satisfied.
  \item[\ref{inv:unseenunconfirmed}] By further restricting all-quantification
    and non-existance quantification.
  \item[\ref{inv:ledgerord}] By condition that $\gl^r.M(\party) \prec
    \gl^r.\ledgerstate$.
\end{enumerate}

\paragraph{Case {\normalfont$\msg{prove}{x, w}$}}

In the ideal world, this query is handled by the simulated functionality
$\simul.\fnizk$. If $(x, w) \notin \lang$ the call returns immediately
with $\bot$ in both worlds, and no variables are modified, giving the
environment no information to distinguish, and preserving $\boldsymbol I$.
Otherwise, the adversary is immediately queried with $\msg{prove}{x}$ in both
worlds. Again, at this point no variables have been modified, preserving
$\boldsymbol I$, and the information handed to the adversary is the same in both
worlds, giving the environment no information to distinguish. The adversary will
eventually respond with a proof $\nizkproof$, which is verified against
constraints in both worlds, and randomly sampled if it does not meet them. By
\ref{inv:proofs}, the constraints are identical in both worlds. Finally $\proofs$
and $W$ are set, and $\nizkproof$ returned in both worlds, giving no
distinguishing information to the environment. The following parts of the
invariant are affected and preserved:

\begin{enumerate}
  \item[\ref{inv:proofs}] By equal update.
  \item[\ref{inv:wit}] By equal insertion into $\fnizkr.W$, and
    $\simul.\fnizk.W$.
  \item[\ref{inv:honestwit}] By relaxing the constraint.
  \item[\ref{inv:unseenunconfirmed}] As the possible results of executing
    transactions consisting of unrecorded statement/proof pairs is constrained
    -- the environment can no longer decide if they should be processed or not.
  \item[\ref{inv:execcons}] As in \ref{inv:unseenunconfirmed}.
  \item[\ref{inv:postunrecorded}] As in \ref{inv:unseenunconfirmed}.
  \item[\ref{inv:iswit}] As only members of $\lang$ are recorded.
  \item[\ref{inv:recordedunconf}] As in \ref{inv:unseenunconfirmed}.
\end{enumerate}

\paragraph{Case {\normalfont$\msg{verify}{x, \nizkproof}$}}

The flow for verification is only slightly more complex than that for proving.
At a high level, the adversary may be given a chance to produce a last-moment
witness for the statement being verified. If it refuses to do so, the proof is
recorded as definitively invalid. We consider three sub-cases: 1. The
statement/proof pair is recorded as either valid or invalid. 2. The adversary
returns a valid witness. 3. The adversary does not return a valid witness.

In Case~1, $\msg{verify}{}$ returns the same value in both worlds by
\ref{inv:proofs}, giving the environment no means to distinguish. Case~2 is
equivalent to the adversary first sending a $\msg{prove}{}$ query for the given
statement, and supplying it with the corresponding proof, and then running the
$\msg{verify}{}$ query. We therefore refer to Case~1, and the case of
$\msg{prove}{}$. In Cases~1 and 2, no state is changed, preserving $\boldsymbol
I$. Finally, for Case~3, $\fnizkr.\disproofs$ is updated equally in both worlds,
and $\bot$ is returned in both worlds, giving the environment no information to
distinguish. In this case, the following parts of the invariant are affected and
preserved:

\begin{enumerate}
  \item[\ref{inv:proofs}] By equal update.
  \item[\ref{inv:rejected}] By relaxing the condition on
    $\fnizkr.\disproofs$.
  \item[\ref{inv:unseenunconfirmed}] As the possible results of executing
    transactions consisting of unrecorded statement/proof pairs is constrained
    -- the environment can no longer decide if they should be processed or not.
  \item[\ref{inv:execcons}] As in \ref{inv:unseenunconfirmed}.
  \item[\ref{inv:postunrecorded}] As in \ref{inv:unseenunconfirmed}.
  \item[\ref{inv:recordedunconf}] As in \ref{inv:unseenunconfirmed}.
  \qed
\end{enumerate}
\end{proof}

As the environment cannot with non-negligible probability between the real and
ideal world in any single action if $\boldsymbol I$ is preserved, and as
$\boldsymbol I$ is preserved with overwhelming probability across each action by
the environment, and holds at protocol initialization, we conclude that the
environment cannot distinguish in the UC game.%
%
%%% Local Variables:
%%% mode: latex
%%% TeX-master: "../main"
%%% End:
