\section{Enforcing Private State Consistency}
\label{sec:pstateconsistency}

The protocol presented so far allows an adversary to arbitrarily set his own
private state. Often it may be desirable to ensure that parties must follow the
rules of the contract, even when it comes to the private state, however. This is
possible, although it also introduces extra costs, and has the caveat of not
functioning with nondeterminism.

The core idea is to store commitments to private states within the public state
of the contract. The contract itself can then verify that the private state is
consistent with this commitment, update it, and then re-commit to the new state,
proving the correctness of every step along the way. Clearly this adds more work
to be verified about the contract, however a more worrying change is that again
the contract needs to be able to process the entirety of the private contract
state. Fortunately using slightly more complex updateable cryptographic
datastructures, such as Merkle trees, can mitigate this problem -- although it
cannot be eliminated entirely, as computation which aggregates the entire
private state will still be as costly.

%%% Local Variables:
%%% mode: latex
%%% TeX-master: "main"
%%% End:
