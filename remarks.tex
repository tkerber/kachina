\section{Conclusion}

We have shown in this paper how to build a large class of smart contracts with
only zero-knowledge and distributed ledgers, and outline how this can be used
and extended upon. To do so we have modeled formally what smart contracts with
privacy are, represented as a state transition function that is fed inputs from
a ledger, and a leakage function that decides what parts of the input are
visible on this ledger. We have then defined which class of such contracts we
will consider in this paper, and presented a protocol, $\kachina$, to construct
them. This protocol utilizes
non-interactive zero-knowledge proofs and state oracles to achieve the desired
smart contract behavior while leaking only part of the computation performed.

While the designs are largely theoretical and detached from any actual
implementation, we stress that they were designed with real-life constraints in
mind: The use of state oracles allows moving most computationally hard, or
storage intensive operations outside of the NIZK itself, reducing their cost.
While the NIZK must still be universal, zero-knowledge constructions with
universal reference strings exist~\cite{CCS:MBKM19}, and are practical to use in our
setting, although they have not yet been proven in the UC model.

In ending this paper, we would like to make clear that this problem space is by
no means solved. We have shown how to realize a specific class of
privacy-preserving smart contracts, however privacy is not such a simple issue
to be addressed by a single paper. In \iffull\autoref{sec:metaparties}\else\cite[Appendix~I]{fullversion}\fi, we sketch
the relation of trust models with privacy, and we believe this taxonomy of
trust, and how each level can be addressed, formalized, and brought into a
unified model, is a crucial long-term research question for providing meaningful
privacy to smart contract systems.

%%% Local Variables:
%%% mode: latex
%%% TeX-master: "main"
%%% End:
