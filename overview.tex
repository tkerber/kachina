\section{Technical Overview}
\label{sec:overview}

We first informally establish the goals and core technical ideas of this
paper. These will be fleshed out in the remainder of the paper's body, with some
of the technical details -- primarily in-depth UC constructions and proofs -- in
the\ \iffull appendix\else technical report~\cite{fullversion}\fi.
We will discuss each of our contributions in turn, and discuss how,
combined, they present a powerful tool for constructing privacy-preserving smart
contract systems.

\paragraph{Our model}

We model smart contracts as \emph{reactive state machines}, which users interact
with by submitting transactions to a distributed ledger. A user submits a
transaction, with the intention to issue some high-level command to the
smart contract, e.g. to cast a vote, or withdraw funds. Once the
transaction is confirmed by the distributed ledger, the user obtains information
about the results of this high-level command: both whether it has been processed, and any
information it may have computed using the contract's state.

As multiple users can interact with the same smart contract system concurrently,
users cannot always predict the effect of their actions; a vote may end before
a user's voting transaction is processed, for instance. As a result the user
may not be able to predict the outcome of the command, or even if it can be
carried out.

To capture privacy, the act of creating a transaction to post on the
distributed ledger is the only point at which we permit privacy leakage. As a
user may go offline at any point, any private information they reveal -- a bid
during an opening phase of an auction for instance -- must be revealed in the
on-chain transaction itself. Formally, we model this with a \emph{leakage
  function} $\lkgfn$, which describes what information is leaked if a user,
seeing a specific contract state, issues a specific command. This function can
also fix choices that an interaction may make -- for instance if the command is ``send a coin
to Bob'', it may decide \emph{which} coin to send to Bob. To give users full
control over their privacy, even when these decisions are complex or randomized,
we ask them to sign off on a description of the leakage before the transaction
is broadcast.\ The leakage in \kachina\ captures
  information which a user purposely decides to reveal, as the functionality
  they gain by doing so is worth whatever damage they take to their private
  information. It is further worth noting that nothing prevents a malicious
  contract from finding clever ways to leak information without being
  observable. This highlights the importance of interacting only with
  trustworthy contracts, and the importance of the leakage descriptor being accurate.

Similarly to the leakage function, the semantics of the contract itself are
largely dictated by a \emph{transition function} $\tnsfn$. It describes how the
state of a smart contract evolves given a command and a few auxiliary inputs
(such as the choice of coin alluded to above).

\paragraph{The core protocol idea}

The \kachina\ core protocol restricts itself to contracts which divide
their state into a public state $\sigma$, and, for each party $\party$, a
private state $\rho_\party$. These correspond to the shared ledger, and a
party's local storage respectively. Transition functions are over pairs
$(\sigma, \rho_\party)$ instead of over \emph{all} private states -- a party may
only change their own private state. Honest users maintain their own private
state in accordance with the contracts' rules, while the contract must
anticipate that dishonest parties may set it arbitrarily (this can be
circumvented by committing to private states, as descripted in
\iffull\autoref{sec:pstateconsistency}\else\cite[Appendix~G]{fullversion}\fi, although it comes at the cost of increased
public state sizes, and loss of anonymity).

A natural construction to achieve privacy in smart contracts utilizing
zero-knowledge proof systems is apparent: On
creating a transaction, a user $\party$ evaluates the transition function against the
current contract state $(\sigma, \rho_\party)$, resulting in a state $(\sigma',
\rho'_\party)$. He creates a zero-knowledge proof that $\sigma \mapsto \sigma'$
is a valid transition of public states (i.e. there exists a corresponding
private state and input such that this transition takes place), and posts the
proof and transition as a transaction.
Locally, the user updates his private state to $\rho'_\party$.

We can also clearly describe the leakage of this sketched protocol: The
transition $\sigma \mapsto \sigma'$ is precisely the information which is
revealed!

\paragraph{State oracles}

The core protocol sketched above has two major problems:
\begin{enumerate}
  \item Due to each transaction containing a proof of transition from one state to another,
    concurrent transactions will almost certainly fail once the state is changed.
  \item The size of the statement being proved, and therefore the size of transactions,
    grows linearly with the overall size of the contract's state.
\end{enumerate}

These drawbacks are especially notable in systems with many users and a high
frequency of transactions: On Ethereum a transaction is almost
certainly applied after many other transactions the author never knew about, nor
should need to know about. The state the contract will be in once it executes
a transaction, is something the transaction's author cannot predict accurately. In the naive system
proofs only succeed in the state they were originally created for, as
\autoref{fig:transferabilitybad} suggests. Instead of capturing a transition
from $\sigma \mapsto \sigma'$, we would rather want to capture a (partial)
function from states to successor states.

\subimport*{fig/}{transferabilitybad}

To solve these issues, we add a layer of indirection for accessing and updating
contract states: Instead of the state being a direct input to the transition function, the
contract has access to \emph{oracles} operating on the public and private
states. The contract makes queries to these oracles: functions which update
the state, and return information about it. To prove the interaction with the
public state correct, users capture the queries they made, and
the responses they expect, in a sequence $((q_1, r_1), \ldots, (q_n, r_n))$: a
\emph{transcript} of oracle interactions. The user proves that, given the
responses expected, they know an input which will make this series of queries.

Conversely, a user validating this transcript can verify this proof, and
evaluate the queries in turn against the public state, ensuring the responses
match. This defines a partial function over public states, which is defined
wherever the responses recorded in the transcript match the results obtained by
evaluating the queries on the current state.

Selecting what queries a contract makes provides a great deal of control over
the domain of the function: a query which has an empty response will always
succeed! In limiting queries to returning only essential information, many
conflicts can be avoided. Transcripts can also be concise about what changes are
made, assuming the queries are encoded in a sufficiently succinct language, such
as most Turing-complete languages.

While not all conflicts are resolved through this as the responses may not match
those expected, it allows the proof to focus on the \emph{relevant} parts of the
state, being compatible with more concurrent transactions, as pictured in
\autoref{fig:transferabilitygood}.

\subimport*{fig/}{transferabilitygood}

In order to be able to model partial transaction success, which
is crucial for modeling transaction fees, we allow for a special query to
be made, $\msg{commit}{}$. $\msg{commit}{}$ queries mark checkpoints in a
transaction's execution, such that if an error occurs after it, the execution up
to this point is still meaningful. This effectively partitions the transcript
into atomic segments. We primarily use this to construct
transaction fees within a smart contract itself, the details of which can be
seen in \iffull\autoref{sec:fees}\else\cite[Appendix~J.5]{fullversion}\fi.

\paragraph{High-level usage}

Even when using state oracles, this protocol is limited to contracts which have
their state fit neatly into accessing only shared public state, and local
private state. The natural description of many contracts does not match this.
For instance: a private currency contract is most directly described through
a \emph{shared private} state tracking the balances of all parties.

However, it \emph{is} simple to express the Zerocash~\cite{SP:BCGGMT14} protocol in terms of
interactions with shared public, and local private states. This provides a
practical means to \emph{achieve} what we can \emph{describe} using a shared
private state. It is important to have both the most natural description of a
contract, and the realization. The former provides a good understanding of
the features and security properties of a contract, while the latter
realizes it.

This idea is nothing but the notion of simulation-based security itself! We use
multiple stages of UC-emulation: First moving from our objective
contract (a private payments contract) to a contract within the \kachina\
constraints on state (a Zerocash contract), and second moving on to the
\kachina\ core protocol. Due to the transitivity of UC emulation, we may therefore
use this ``\kachina\ method'' to construct the objective of private payments. This
process is outlined in \autoref{fig:kachinamethod}.

Our model is designed to facilitate this usage. Specifically for
modeling objective contracts the model allows the adversary to provide an
additional adversarial input to each transaction. This input allows the
simulator to control some parts of the ideal behaviour similar to the simulator's
influence on an ideal functionality, for instance to ensure
ideal world addresses match real-world public keys.

%%% Local Variables:
%%% mode: latex
%%% TeX-master: "main"
%%% End:
