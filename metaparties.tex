\section{Meta-Parties and Their Relation to Alternative Trust Models}
\label{sec:metaparties}

So far we have presented, and argued for, a model of smart contracts with clear
black-and-white privacy: Users have their own perfectly private local state, and
access to a perfectly public shared state. While we believe this to be the best
starting point for approaching the issue of privacy in smart contracts, reality
is not so simple: Often users have more complex relations with each other.

To consider this more carefully, we can consider that any piece of data in a
smart contract must have a set of \emph{owners} $\mathbb{O}$, who can interact
with it. Furthermore, in any real system, there are parties which can, together,
decipher the actual data itself, and break the privacy of it. Let us refer to
the set of all combinations of parties able to decipher the data as
$\mathbb{T}$. While not strictly necessary, in general it is reasonable to
assume that the owners are also the users able to break privacy, that is
$\mathbb{T} \subseteq 2^{\mathbb{O}}$. While clearly there are many possible
combinations here, a few stand out as interesting, and we observe that they all
relate to some interpretation of privacy-preserving smart contracts:

\begin{itemize}
  \item $\mathbb{O} = \parties$, $\mathbb{T} = 2^\parties$: This is the
    setting of Ethereum, and of the $\sigma$ used in this work. Data is public,
    but can be interacted with by all.
  \item $\mathbb{O} = \set{\party}$, $\mathbb{T} = \set{\party}$: This is the
    setting of $\rho$ used in this work. Data is private, but cannot be
    interacted with by anyone else.
  \item $\mathbb{O} = \parties$, $\mathbb{T}$ is all subsets of $\parties$ with
    a resource majority (regardless of work, stake, or what other honest
    majority assumption is being made): This setting is feasible by running MPC
    across the honest majority of the underlying consensus protocol.
  \item $\mathbb{O}$ is a fixed-size set, $\mathbb{T} = \set{m}$: This is the
    setting of Hawk~\cite{SP:KMSWP16}, in which a single party is trusted with privacy.
  \item $\mathbb{O}$ is a fixed-size set, $\mathbb{T} = \{\mathbb{O}\}$: This is
    the setting of privacy-preserving state-channels, in which parties run MPC
    out-of-band to agree on updates.
  \item $\mathbb{O}$ is a fixed-size set, $\mathbb{T} = 2^{\mathbb{O}}$: This is
    the setting of public state-channels, in which parties run Arbitrum-like
    protocols out-of-band to agree on updates.
\end{itemize}

In particular, this work only directly concerns itself with the first two of
these. It is clear, however that different problems call for different
solutions, and ideally a smart contract system would encode all of these trust
systems, not just one, or a few. Part of the reason for the choice of the first
two is that they are sufficient for constructing the rest, being the exteremes
of the spectrum.

The case of Hawk, for instance, was already described above in
\autoref{sec:nonatomic}. We will sketch how state channels might be
modeled on top of \kachina, although we stress that a full formal treatment of
this, and other settings, will be left for future work.

A state channel between two users can be interpreted as the two users
constituting a ``metaparty'' -- a single entity consisting of multiple parties.
This is subject to some access control for when the constituent parties can act
on behalf of both -- commonly requiring agreement from all constituent parties.
If Alice and Bob open a new state channel, this can be seen as creating a new
combined party of $(\text{Alice}, \text{Bob})$.

In \kachina, this party again has its own private state, and for state channels,
this can track the most recent update of the channel. Updates are now operations
that only affect the private state of this combined party, and as argued in
\autoref{sec:nonatomic} can be left off the ledger entirely. Interestingly,
the access structure for closing channels, and reading the current state is more
permissive in most state or payment channels -- requiring only one user to
initiate it.

Given a state channel system, most of it can be implemented in \ankachina\ smart
contract. It is not new for state or payment channels to use smart contracts,
however this is typically only for the opening and closing of the channel.
We observe that in \kachina\ the update of the channel can also be modeled.

This approach of metaparties is useful, but not optimal. For instance, a
contract cannot interact with both Alice's private state, and the state channel
between Alice and Bob at the same time, as presented here. Further, how the
constituents of a metaparty reach consensus on whether an action is permitted or
not is unclear, and varies from case to case. We leave as future work giving
first-class treatment to data owned by multiple parties.

%%% Local Variables:
%%% mode: latex
%%% TeX-master: "main"
%%% End:
