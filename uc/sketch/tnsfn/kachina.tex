\begin{transitionfnsketch}{$\tnsfnmainkachina(\tnsfnkachina)$}
  \vspace{-1em}
  \begin{receiveinput*}{$((\sigma, \boldsymbol\rho), \party, w,
      (\transcript_\sigma, z), \cdot)$}
    \State \Let $(\partstate{\sigma}, \transcript_\sigma', \partstate{\rho}, \cdot, \partresults) \gets
      \runtnsfn(\sigma, \boldsymbol\rho[\party], w, z, \party \in \honest)$
      \State \Let $y \gets \bot; C \gets \top$
      \State \Let $\splittranscript \gets
        \vn{split}(\transcript_\sigma, \msg{commit}{}); \splittranscript' \gets
        \vn{split}(\transcript_\sigma', \msg{commit}{})$
    \For{$(\transcript_i, \transcript_c, \sigma', \rho', y') \loopin \vn{zip}(\splittranscript,
      \splittranscript', \partstate{\sigma},
        \partstate{\rho}, \partresults)$}
      \If{$\sigma' = \bot \lor \rho' = \bot \lor \transcript_i \neq \transcript_c$}
        \State \Let $C \gets \bot$
        \State \Break
      \EndIf
      \State \Let $\sigma \gets \sigma'; \boldsymbol\rho[\party] \gets \rho'; y
        \gets y'$
    \EndFor
    \State \Return $((\sigma, \boldsymbol\rho), C, y)$
  \end{receiveinput*}

  \noindent Where $\runtnsfn(\sigma, \rho, w, z, \cdot)$ runs
  $\tnsfnkachina_{\oracle_\sigma, \oracle_\rho}(w)$, and returns $(\partstate{\sigma},\allowbreak
  \transcript_\sigma,\allowbreak \partstate{\rho}, \transcript_\rho, \partresults)$ (see
  \iffull\autoref{sec:fullfuncts}\else\cite[Appendix~C]{fullversion}\fi\ for a full specification).
\end{transitionfnsketch}

%%% Local Variables:
%%% mode: latex
%%% TeX-master: "../../../main"
%%% End:
