\subsection{State Oracles and Transcripts}
\label{sec:oracles}
\label{sec:context}
\label{sec:auctionexample}

We introduce \emph{state oracles} and \emph{state oracle transcripts} to
abstract interaction with a contract's state. We choose this abstraction
primarily for its flexibility, and many other approaches are possible, such as
byte-level memory accesses, or specific data structures such as set of unspent
transactions. These can be seen as instances of state oracles. We make use
  of the notation $[a, b, c]$ to denote a list of $a$, $b$, and $c$, with the
  concatenation operator $\concat$, and the empty list $\epsilon$. We
  use the function $\vn{last}$ to retrieve the last element of a list, and
  $L[i]$ to denote the $i$th element of the list $L$.

\paragraph{An example}

To better motivate the need to abstract interactions with a contract's state, we
will use a representative example smart contract, and discuss how different
abstractions of its state will affect it.

Our example is a \emph{sealed bid auction} contract\footnote{This contract is
  designed to make a good example, not a good auction -- we do not recommend
  using it as presented.}, which we assume has some means of telling the time,
and\ interacting with two on-chain assets, one public and one private. These may be
  constructed similarly as in \autoref{app:privatepayments}, however should be
  holdable and spendable by other contracts. We do not go into detail of this
  construction; this idea is fleshed out in detail in Zether~\cite{zether}.
The auction is opened by the \emph{seller} party, and multiple \emph{buyer}
parties may bid on it. The auction has three stages: Bidding, opening, and
withdrawing. The auction contract allows for the following
interactions:

\begin{itemize}
  \item At initialization, the seller transfers ownership of the public asset $A$ to the
    auction contract.
  \item In Stage 1, buyers submit their bids, transferring some amount of
    the private asset $B$ to the auction contract, which remains anonymous.
  \item In Stage 2, buyers \emph{reveal} their bid. If the
    buyer's bid exceeds the currently maximum revealed bid, they reveal their
    committed asset, increase the maximum bid, and they record themselves as the
    winning bidder. Otherwise, they withdraw their bid from the contract without
    revealing its value.
  \item In Stage 3, buyers withdraw any assets they own after the
    auction -- either their (losing) bids, or the sold asset (for the highest
    bidder). The seller withdraws the highest bid, or the original asset if no
    bids were made.
  \item In Stage 1 and 2, the seller may advance the stage.
\end{itemize}

\noindent This contract needs to maintain in its state:
\begin{itemize}
  \item The current stage the auction is in.
  \item A reference to the asset being sold.
  \item A set of bids made.
  \item The winning bid, its value, and who made it, during the reveal phase.
  \item A set of losing bids, which have not yet been withdrawn, during the
    reveal phase.
  \item Privately, a user remembers which bids are theirs, and how to reveal
    them.
\end{itemize}

Suppose we adopted a naive approach to state transitions, and proved the
transitioning from one state to another directly, with no abstraction of any
kind. During the bidding phase it is easily possible for multiple users to
attempt to bid simultaneously (especially considering the delay until
transactions become confirmed by an underlying ledger). In this case, only one
of these transactions will succeed -- as soon as this transaction changes the
state by adding its own bid, the proof of any other simultaneous transaction
becomes invalid.

The simple abstraction of byte-level access would allow a buyer and a seller
to withdraw concurrently, as their withdrawals affect different parts of the
state. It does not do so well in allowing concurrent bids to be made, however.
If the set is implemented with a linked list, for instance, two users attempting
to add their own bid simultaneously will change the same part of the state: the
pointer to the next element.

A smart abstraction should realize that whichever user bids first, the resulting
set of bids is the same, even if its binary representation may not be. Even if
the order of the interactions matters, a smart abstraction may allow concurrent
interactions. When claiming the maximum bid in the auction, Alice may increase
it to 5, while Bob may increase it to 7 concurrently. It should not matter to
Bob's transaction if the maximum bid is currently 3, or 5 -- although Alice's
must be rejected if the bid is increased to 7 first.

\paragraph{General-purpose state oracles}

The abstraction we propose is that of \emph{programs}. Appending a value to an
linked list can be encoded as a program which a) traverses to the end of the
current list, b) creates a new cell with the input value, and c) links this from
the end of the list. Formally, these programs are executed by a universal
machine called a \emph{state oracle} with access to the current (public or
private) state $\arbstate$, and potentially an additional \emph{context} $z$.

\begin{definition}
  \label{def:oracle}
  A state oracle $\oracle = \utm(\arbstate_0, z)$, given an initial state
  $\arbstate_0$, and context $z$, is an interactive machine internally
  maintaining a state $\arbstate$, a transcript $\transcript$, and a vector
    of fallback states $\partstate{\arbstate}$\ (initially set to the input
  $\arbstate_0$, $\epsilon$, and $[\arbstate_0]$\ respectively), which
  permits the following interactions:
  \begin{itemize}
    \item Given a $\msg{commit}{}$ query, set $\partstate{\arbstate} \gets
        \partstate{\arbstate} \concat [\arbstate]$, and append $\msg{commit}{}$ to
        $\transcript$.
    \item Given a query $q$ while $\arbstate$ is $\bot$, return $\bot$.
    \item Otherwise, given a query $q$, compute $(\arbstate', r) \gets q(\arbstate, z)$.
      Update $\arbstate$ to $\arbstate'$, append $(q, r)$ to $\transcript$, and
      return $r$.
    \item $\vn{state}(\oracle)$ returns $(\partstate{\arbstate} \concat [\arbstate], \transcript)$.
  \end{itemize}
\end{definition}

The context $z$ is empty ($\varnothing$) for state oracles operating on the public state, and is used in
state oracles operating on the private state for fine-grained read-only access
to the state during transaction creation, e.g. to allow private state oracles to read the public
state.\ 
Specifically, the oracle operating on the private state can read both the public
and private states for: a) the confirmed state at the time the transaction was
created ($\sigma^o$ and $\rho^o$), and b) the \emph{projected} state, an
optimistic state in which all of the user's unconfirmed transactions are
executed, at the time the transaction was created ($\sigma^\pi$ and $\rho^\pi$).
This can be used to make sure new transactions do not conflict with pending ones:
Selecting which coin to spend uses the confirmed state to ensure the coin
\emph{can be spent}, and the projected state to ensure a coin is not
\emph{double spent}. The context is also used to provide a source of randomness
$\eta$ to the private state oracle. In total, the context of the private state
oracle is $(\sigma^o, \rho^o, \sigma^\pi, \rho^\pi, \eta)$. The context to the
public state oracle is empty ($\varnothing$), and we will sometimes omit it.

We say that the oracle \emph{aborts} if it sets its state to $\bot$. The
  state will then be rolled back to a safe point, specifically the last
  $\msg{commit}{}$ where the state was non-$\bot$. Looking forward, we will decompose the
transition function $\tnsfn$ into three components: An oracle operating on the
public state $\sigma$, an oracle operating on $\party$'s private state
$\rho_\party$, and a ``core'' transition function $\tnsfnkachina$. This process
is described in detail in \autoref{sec:classkachina}, with an overview of the
interactions of $\tnsfnkachina$ with public and private state oracles given
in \autoref{fig:oracleflow}.

\subimport*{../fig/}{oracleflow}

The notion of \emph{oracle transcripts} is crucial in the functioning
of \kachina, as it provides a means to decouple the part of a transaction which
is proven in zero-knowledge from both the public and private states entirely: We
only prove that given some input, and a sequence of responses recorded in the
public state transcript, the smart contract must have made the recorded queries.

\paragraph{Revisiting our example}

As an illustration, we show how our auction example interacts with state
oracles. We define the auction's states more precisely first, where users are
identified by public keys, denoted with $\pk$:

\newcommand{\openbid}{\vn{bidOpen}}
\newcommand{\bidcomm}{\vn{bidComm}}

\begin{itemize}
  \item The current stage, $\vn{stage} \in \set{1, 2, 3}$.
  \item A reference to the asset being sold and who is selling it: $a, \pk_s$.
  \item A set of bids made $S$.
  \item The winning bid, its value, and who made it: $b,
    v, \pk_b$.
  \item A set of not yet withdrawn losing bids: $R$.
  \item Privately, a user remembers openings to their bids, the committed bid
    itself, and its value: $\openbid, \bidcomm, v$.
\end{itemize}

Overall, the public state is defined as $\sigma \eqdef (\vn{stage}, \pk_s, a, b,
\allowbreak v,\allowbreak \pk_b, S, R)$, and the
private state is defined as $\rho \eqdef (\openbid,\allowbreak \bidcomm, v)$. The public state is
initialized by the seller to $(1,\pk_s, a, \varnothing, 0, \varnothing,\allowbreak
\varnothing, \varnothing)$.

The oracle queries corresponding to each interaction with the contract are
given as closures, i.e. sub-functions which make use of some of their
parents local variables. To clarify where this is the case, we place such
variables in the subscript of the function name. These functions are passed to
either the public or private state oracle as the input $q$, as specified in
\autoref{def:oracle}.

\begin{itemize}
  \item Bidding: Given a asset opening $\openbid$, with value
    $v$, corresponding to an asset commitment $\bidcomm$, which has been
    bound to the auction contract, $\tnsfnkachina$ first makes the following public
    oracle query:
    \begin{algorithmic}
      \Function{$\vn{makeBid}_{\bidcomm}$}{$(\vn{stage}, \pk_s, a, b, v, \pk_b, S, R)$}
        \State \Assert $\vn{stage} = 1$
        \State \Return $((\vn{stage}, \pk_s, a, b, v, \pk_b, S \cup \set{\bidcomm}, R),\top)$
      \EndFunction
    \end{algorithmic}
    Further, it makes the following private oracle query:
    \begin{algorithmic}
      \Function{$\vn{recordBid}_{\openbid, \bidcomm, v}$}{$\cdot, \cdot$}
        \State \Return $((\openbid, \bidcomm, v),\top)$
      \EndFunction
    \end{algorithmic}
  \item Revealing: Given a public key to redeem the funds to in case of losing
    the auction, $\tnsfnkachina$ first makes a private oracle query to retrieve which bid is owned:
    \begin{algorithmic}
      \Function{$\vn{retrieveBid}$}{$(\openbid, \bidcomm, v), \cdot$}
        \SplitState{\Return $((\openbid, \bidcomm, v),\allowbreak (\openbid, \bidcomm, v))$}
      \EndFunction
    \end{algorithmic}
    Next, the contract makes a further private oracle query for the expected maximum bid,
    to determine if the buyer's bid is higher:
    \begin{algorithmic}
      \Function{$\vn{projMax}$}{$\rho, z = (\cdot, \cdot, \sigma^\pi =
        (\ldots, v', \ldots), \cdot, \cdot)$}
        \State \Return $(\rho, v')$
      \EndFunction
    \end{algorithmic}
    If this query returns $v' < v$, the contract attempts to claim the maximum
    bid with the public oracle query\footnote{Note that the claim may fail if
      the maximum bid increased from the one projected at the time of
      transaction creation.}:
    \begin{algorithmic}
      \Function{$\vn{claimMax}_{\openbid,\bidcomm,v,\vn{pk}}$}{$\sigma$}
        \State \Let $(\vn{stage}, \pk_s, a, b_o, v_o,\allowbreak \pk_o, S, R)
          \gets \sigma$
        \State \Assert $\bidcomm \in S \land v > v_o \land \vn{stage} = 2$
        \SplitState{\Return $((\vn{stage}, \pk_s, a, \openbid, v, \pk, S \setminus
          \{\bidcomm\},$\splitstatebreak$R \cup \set{(b_o, \pk_o)}),\top)$}
      \EndFunction
    \end{algorithmic}
    If the original value test fails, on the other hand, instead
    the contract transfers the ownership of $\bidcomm$ via the underlying asset
    system to $\pk$, and runs the public oracle query:
    \begin{algorithmic}
      \Function{$\vn{claimLoss}_{\bidcomm}$}{$(\vn{stage}, \pk_s, a, b, v_o, \pk_o,
        S, R)$}
        \State \Assert $\bidcomm \in S \land \vn{stage} = 2$
        \SplitState{\Return $(\top, (\vn{stage}, \pk_s, a, b, v_o, \pk_{o}, S \setminus \set{\bidcomm},\allowbreak
          R))$}
      \EndFunction
    \end{algorithmic}
  \item Withdrawing: Given a public key $\pk$, which the caller knows the
    corresponding secret key for, the contract will make an oracle query to
    determine which assets to transfer ownership of, and to un-record them in a
    public oracle query:
    \begin{algorithmic}
      \Function{$\vn{withdraw}_\pk$}{$(\vn{stage}, \pk_s, a, b, v, \pk_b, S, R)$}
        \State \Assert $\vn{stage} = 3$
        \If{$\pk = \pk_s \land b \neq \varnothing$}
          \State \Return $((\vn{stage}, \varnothing, a, \varnothing, \varnothing, \pk_b, S,
          R),(B, b))$
        \ElsIf{$\pk = \pk_b \land a \neq \varnothing$}
          \State \Return $((\vn{stage}, \pk_s, \varnothing, b, v, \varnothing, S, R),(A, a))$
        \ElsIf{$\exists c : (c, \pk) \in R$}
          \SplitState{\Return $((\vn{stage}, \pk_s, a, b, v, \pk_b, S, R \setminus \set{(c,
            \pk)}),\allowbreak (B, c))$}
        \EndIf
      \EndFunction
    \end{algorithmic}
  \item Advancing the stage: The seller (given their public key $\pk$) may
    advance the contracts stage from 1 or 2 to 2 or 3 respectively with a public
    oracle query:
    \begin{algorithmic}
      \Function{$\vn{advanceStage}_\pk$}{$(\vn{stage}, \pk_s, a, b, v, \pk_b, S,
        R)$}
        \State \Assert $\pk = \pk_s \land \vn{stage} \in \set{1, 2}$
        \State \Return $((\vn{stage} + 1, \pk_s, a, b, v, \pk_b, S, R), \top)$
      \EndFunction
    \end{algorithmic}
\end{itemize}

This example does not handle corner cases (such as buyers bidding multiple
times), and is not intended for practical use. Instead, its purpose is to
illustrate the advantages state oracles provide: The query an interaction will
make, and the response it will receive, are often not affected by other
interactions. Concurrent bids do not conflict, for instance. The representation
of data is also not crucial, as the state oracles may themselves interact with
abstract data types.

We complete our example by specifying the core transition function $\tnsfnkachina$,
under the assumptions that a means to call into a separate asset management
system (a contract that permits transferring ownership of assets between public
keys), such as presented in \iffull\autoref{sec:payments}\else\cite[Appendix~J.4]{fullversion}\fi, exists. We also assume that
a user's public key can be retrieved with a shared ``identity'' contract.

\subimport*{../uc/tnsfn/}{auction}


\paragraph{Using transcripts}
\kachina\ relies on a few key observations on how transcripts relate to the
original state oracle execution. To begin with, we define a few ways in which
transcripts may be used.
\begin{definition}
  A state oracle transcript $\transcript$ may be applied to a state $\arbstate$
  in a context $z$. We write $\partstate{\arbstate} \gets \transcript(\arbstate, z)$, or if
  $z = \varnothing$, $\partstate{\arbstate} \gets \transcript(\arbstate)$, the operation of
  which is defined through the following loop:
  \begin{algorithmic}
    \Function{$\transcript$}{$\arbstate, z$}
      \State \Let $\oracle \gets \utm(\arbstate, z)$
      \For{$(q_i, r_i) \loopin \transcript$}
        \State \SendAndReceive*{$q_i$}{$\oracle$}{$r$}
        \If{$r \neq r_i$}
          \State \Let $(\partstate{\arbstate} \concat \cdot, \cdot) \gets \vn{state}(\oracle)$
          \State \Return $\partstate{\arbstate} \concat \bot$
        \EndIf
      \EndFor
      \State \Let $(\partstate{\arbstate}, \cdot) \gets \vn{state}(\oracle)$
      \State \Return $\partstate{\arbstate}$
    \EndFunction
  \end{algorithmic}
  If a transcript is malformed, applying it will result in $[\arbstate,
    \bot]$.
\end{definition}

If $\arbstate' = \transcript(\arbstate, z) \neq \bot$, then 
The returned $\partstate{\arbstate}'$ is indistinguishable from the internal state
$\partstate{\arbstate} \concat [\arbstate]$ of the state oracle $\utm(\arbstate_0,
z)$, given the same sequence of queries. This allows users to replicate the
effect other users' queries have on the public state, without knowing \emph{why}
these queries were made.

\begin{definition}
  A sequence of transcripts and contexts $X = ((\transcript_1, z_1), \ldots,
  (\transcript_n, z_n))$ is applied by applying each transcript in order. We write
  $\transcript^*_X(\arbstate)$, which has the recursive definition:
  \begin{itemize}
    \item $\transcript^*_\epsilon(\arbstate) \eqdef \arbstate$
    \item $\transcript^*_{[(\transcript, z)] \concat X}(\arbstate) \eqdef
      \vn{last}(\transcript^*_X(\transcript(\arbstate, z)))$
  \end{itemize}
\end{definition}

We will need to use a transcript in place of a state oracle, replicating the
oracle behaviour without access to the state, provided it receives the same
sequence of request. We refer to this as a \emph{transcript oracle}:

\begin{definition}
  A transcript $\transcript = ((q_1, r_1), \ldots, (q_n, r_n))$
  (potentially including $\msg{commit}{}$ messages)\ induces a
  transcript oracle $\oracle(\transcript)$, which behaves as follows:
  \begin{itemize}
    \item Recorded $\msg{commit}{}$ messages are ignored.
    \item When a query $q_i'$ is made as the $i$th query to the transcript
      oracle, return $r_i$ if $q_i' = q_i$, otherwise abort by returning $\bot$
      for this, and all subsequent queries.
    \item We define the predicate $\vn{consumed}(\oracle)$ as $\top$ if exactly $n$
      queries were made, and $\bot$ otherwise.
  \end{itemize}
  If in an interaction with the oracle, $\vn{consumed}$ holds, the transcript
  was \emph{minimal} for this interaction.
\end{definition}

If the transcript oracle $\oracle(\transcript)$ doesn't abort when used as an
oracle in some function, then it behaves identically to the original universal
oracle that was used to generate the transcript. We use this fact to generate
zero-knowledge proofs about transactions -- we prove that each oracle query in a
transcript was made, and that the behavior is correct, \emph{given the
  responses the transcript claims}. We also prove that $\vn{consumed}(\oracle)$
holds, ensuring the transcript doesn't just start with the queries an honest
execution would make, but that it matches them exactly.

These are used together to define how a transaction is made, and how it is
applied: Alice generates a transcript for the oracle accesses her transaction
will perform, and proves this transcript both correct and minimal. She sends the
transcript and proof to Bob, who is convinced by the proof of correctness and
minimality, and can therefore reproduce the effect of the transaction by
applying the transcript to the state directly.

%%% Local Variables:
%%% mode: latex
%%% TeX-master: "../main"
%%% End:
