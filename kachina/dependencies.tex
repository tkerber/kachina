\subsection{The Challenge of Dependencies}
\label{sec:deps}

If a transaction $\tx_1$ moves funds from Alice to Bob, and $\tx_2$ moves funds
from Bob to Charlie, the order $\tx_2\ldots\tx_1$ may not be valid, if $\tx_2$
relies on the funds Bob received from Alice. When a dependency like this is
violated in interacting with the public state, attempting to apply the dependent
transaction first will fail, and the transaction is rejected.

How such interactions affect a user's private state is more tricky to handle.
While two different parties cannot conflict with each other on private state
changes due to domain separation, parties may encounter \emph{internal}
dependencies.

A party starting with the private state $\rho_1$, makes a transaction $\tx_1$
which advances their private state to $\rho_2$. Afterwards, they make the
transaction $\tx_2$, their private state ending up as $\rho_3$. If these
transactions are made shortly after each other, $\tx_2$ may be placed before
$\tx_1$ on the ledger. It is possible that $\tx_2$ uses information from
$\tx_1$, such as a secret key, and that it makes no sense without it.

Should a user ignore the reordering, and stick with the state $\rho_3$? This
can introduce inconsistencies between the public state and private state. Should
the user apply the private state transcript of $\tx_2$ and hope for the best --
but risk a catastrophic failure if it cannot be applied? Neither are ideal.
Instead, we propose that $\tx_2$ should publicly declare that it depends on
$\tx_1$, and rely on on-chain validation to ensure they are applied in the
correct order.

If a user has a set of unconfirmed transactions $U$, and is adding the new
transaction $\tx$ in the ledger state, dependencies should ensure
that any permutation of $U \cup \set{\tx}$ results in a consistent interaction
with the user's private state -- i.e. result in a non-$\bot$ private state.
Further, this should even be the case if these transactions are only partially
  successful -- regardless as to which $\msg{commit}{}$ point was reached.

An overeager approach would be to ensure all unconfirmed transactions are
dependencies, and in the order that they were made. With domain separation and
sufficiently abstract interactions it is likely that only few transactions
actually depend on each other. This can be application specific, and to account
for this we allow for contracts to specify a function $\vn{dep}$ to declare
dependencies. We constrain how this function may behave, and provide the
all-purpose fallback of all unconfirmed transactions.

For most practical cases that we have observed, private state oracles do
  not conflict or enter into complex dependencies with each other. Most often,
  their state management is simple: sampling and storing secrets.
  The formal machinery presented in this section is to allow this intuition that
the transactions do not depend on each other to be justified in many cases.

\paragraph{Formal definition}
The formal definition of dependency functions is complex; we begin by
introducing some mathematical notations. In addition to this notation, we
  make use of the following functions: a) the higher-order function $\vn{map}$.
  b) an index function, which returns the index of an element in a list,
  $\vn{idx}$. c) the tuple projection functions $\vn{proj}_i$, which return
  the $i$th element of a tuple. d) the list flattening function
  $\vn{flatten}$, which, given a list of lists, returns a list of the inner
  lists concatenated. e) the function $\vn{take}$, which returns the prefix of a
  list containing a specified number of items. f) the function $\vn{zip}$, which
  combines $n$ lists into a list of $n$-tuples.

\begin{definition}
  For any finite set $X$, $S_X$ is the set of all permutations of $X$, where each
  permutation is a list.
\end{definition}

\begin{definition}
  The \emph{subsequence relation} $X \sqsubseteq Y$ indicates that each element
  of the list $X$ is present in $Y$, in the same order:
  \begin{align*}
    X \sqsubseteq Y \eqdef \;& X \subseteq Y \land 
    (\forall a, b \in X : \vn{idx}(X, a) < \vn{idx}(X, b) \\ &\Longrightarrow
    \vn{idx}(Y, a) < \vn{idx}(Y, b))
  \end{align*}
\end{definition}

\noindent We define an expansion of transactions into useful components: As a
transaction has no private data within it, we use this to refer to this data.

\begin{definition}
  A transcript $\transcript$'s corresponding \emph{commit-sep\-a\-rat\-ed} transcript
  $\splittranscript$ is a list of lists of query/response pairs, corresponding
  to splitting $\transcript$ at each $\msg{commit}{}$. We write
  $\splittranscript = \vn{split}(\transcript, \msg{commit}{})$.
\end{definition}

\begin{definition}
  A secret-expanded transaction is a tuple $(\tau, \splittranscript, z,\allowbreak D)$,
  consisting of the transaction object $\tau$, the commit-separat\-ed private state transcript
  $\splittranscript$, the context $z$, and the dependencies $D$.
\end{definition}

\noindent We define the format of transactions handled by the dependency
function. We make use of ``confirmation depth'', the vector of which is denoted
$\confirmations$. This is a vector of natural numbers, representing how many
parts of the corresponding commit-separated transcript executed successfully.

\begin{definition}
  A list $X$ of secret-expanded transactions' dependencies may be
  \emph{satisfied} given a set of still unconfirmed transaction $U$ and a
    list of confirmation depths $\confirmations$, denoted by $\vn{sat}(X,
  \confirmations, U)$, which is defined formally below. Informally, it
  states that each transaction in $X$ must be preceeded by its dependencies, in
  order, and that each of these dependencies should have executed fully, rather
  than partially.
  \begin{itemize}
    \item $\vn{sat}(\epsilon, \confirmations, U) \eqdef \top$
    \item \sloppy$\vn{sat}(X \concat (\cdot, \cdot, \cdot, D),
      \confirmations \concat \cdot, U) \eqdef \vn{sat}(X, \confirmations, U) \land
      (D \cap U) \sqsubseteq \allowbreak \vn{map}(\vn{proj}_1, X) \land \forall
      d \in D, \splittranscript, z, D', i: (d, \splittranscript, z, D') =
      X[i] \implies |\splittranscript| = \confirmations[i]$
  \end{itemize}

  We write $\vn{sat}^*(X, U)$ as a shorthand for the case where $\confirmations$ are
  maximal: i.e. $\confirmations[i] = |\vn{proj}_2(X[i])|$.
\end{definition}

\fussy

\noindent We define what transcripts will actually be executed for a given
sequence of confirmation levels.

\begin{definition}
  The \emph{effective sequence of transcripts} (denoted $ES(X,
  \confirmations)$), given a list of secret-expanded transactions and a list of
  confirmation depths of equal length, is the sequence of confirmed transcript
  parts, along with their contexts, defined as:

  \noindent
  $ES(X, \confirmations) \eqdef \vn{flatten}(\vn{map}(\lambda ((\cdot,
  \splittranscript, z, \cdot), c) : \allowbreak\vn{map}(\lambda \transcript : \allowbreak(\transcript,
  z),\allowbreak \vn{take}(\splittranscript, c)), \vn{zip}(X, \confirmations)))$
  
  We write $ES^*(X)$ as a shorthand for the case where $\confirmations$ are
  maximal: i.e. $\vn{proj}_i(\confirmations) = |\vn{proj}_2(\vn{proj}_i(X))|$.
\end{definition}

\noindent We define the central invariant the dependencies must preserve:
That the private state can always be advanced.

\begin{definition}
  \sloppy
  \label{def:invdep}
  The dependency invariant $J(X, \rho)$, given a set $X$ of secret-expanded
  transactions, states that any permutation of a subset of $X$'s private state
  transcripts which have their dependencies satisfied can be successfully applied to $\rho$.
  $J(X, \rho) \eqdef \forall Y \subseteq X, Z \in S_Y, \confirmations : \vn{sat}(Z,\allowbreak
  \confirmations, \vn{map}(\vn{proj}_1, X)) \implies
  \transcript^*_{ES(Z, \confirmations)}(\rho) \neq \bot$
\end{definition}

\fussy
\noindent Finally, we define the constraints on the dependency function.

\begin{definition}
  \label{def:dep}
  A dependency function $\vn{dep}(X, \transcript, z)$ is a pure function taking
  as inputs a set of secret-expanded unconfirmed transactions $X$, a new
  private state transcript $\transcript$, and a new context $z$, returning a
  list of transaction objects. It must satisfy the following conditions:
  \begin{enumerate}
    \item If called with non-honestly generated transcripts or contexts, no
      constraints need to hold.
    \item The result must be a subsequence of the transactions in $X$:
      $\vn{dep}(X, \transcript, z) \sqsubseteq \vn{map}(\vn{proj}_1, X)$
    \item When adding a new transaction $\tx$, with the corresponding\
      private state transcript $\transcript$ (where its
        commit-separated form is $\splittranscript$) and context $z$, the
      dependency invariant $J$ is preserved: $\textbf{let }\allowbreak Y =
      X\allowbreak \concat (\tx, \splittranscript,\allowbreak
      z\allowbreak =\allowbreak (\cdot, \rho^o, \cdot, \cdot, \cdot),
      \vn{dep}(X,\allowbreak \transcript,\allowbreak z)) \allowbreak\textbf{ in
      }\allowbreak \transcript^*_{ES^*(Y)}(\rho^o)\allowbreak \neq \bot \land J(X, \rho^o)\allowbreak \implies
      J(Y,\allowbreak \rho^o)$
  \end{enumerate}
\end{definition}

The dependency function $\vn{dep}(X, \transcript, z) = \vn{map}(\vn{proj}_1, X)$
can always be used, as it maximally constraints the possible permutations which
satisfy dependencies.

%%% Local Variables:
%%% mode: latex
%%% TeX-master: "../main"
%%% End:
