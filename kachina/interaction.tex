\subsection{Interaction Between Smart Contracts}

The example in \autoref{sec:oracles}, makes the natural assumption (in the
setting of smart contracts), of being able to interact with other components
-- in this case with an asset system. Most interesting applications of smart
contracts seem premised on such interactions. We consider how multiple
contracts may interact in
\iffull\autoref{sec:loopmux}\else\cite[Appendix~J.3]{fullversion}\fi, however
we stress that a full treatment is left as future work.

In particular, how various contracts can be independently proven secure and
composed in a general system alongside other, potentially malicious contracts,
is not handled in this paper. Instead, where we assume interaction, we limit
ourselves to a closed smart contract system with a small set of non-malicious
contracts, such as the auction contract and the asset system in
\autoref{sec:oracles}.

While it is tempting to delegate such interactions to the native
compositionality and interactiveness of UC, this does not reflect the reality of
smart contract interactions, where the executions of multiple contracts are
atomically intertwined. While related issues of interaction with the
environment have been considered in the literature, for instance in~\cite{AC:CEKKR16,muc},
they do not fully address our scenario, in which multiple branches can be
executed in projection. We therefore believe that studying the interaction and
composition of smart contract transition and leakage functions requires further
work, with this work providing a foundation.

%%% Local Variables:
%%% mode: latex
%%% TeX-master: "../main"
%%% End:
